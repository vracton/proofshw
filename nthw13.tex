\documentclass{article}
\usepackage{enumitem}
\usepackage{fancyhdr}
\usepackage[a4paper, left=1in, right=1in, top=1in, bottom=1in]{geometry}
\usepackage{amsfonts}
\usepackage{amsmath}
\usepackage[many]{tcolorbox}
\usepackage{xcolor}
\usepackage{cancel}
\usepackage{siunitx}
\tcbuselibrary{skins}

\pagestyle{fancy}
\cfoot{}
\lhead{}

\definecolor{seablue}{rgb}{0.0, 0.4, 0.7}
\definecolor{sandpeach}{rgb}{0.98, 0.87, 0.7}


\chead{\Large\bf Number Theory HW \#13 - Rational Numbers}
\lhead{Page \thepage}
\rhead{Sonit Sahoo}
\setlength{\headheight}{18.0pt}

\usepackage{amssymb,xcolor}
\newcommand{\correct}{\textcolor{green}{\checkmark}}
\newcommand{\wrong}{\textcolor{red}{\times}}

\newtcbtheorem[]{problem}{Problem}{
    enhanced,
    colback = sandpeach!8,
    colbacktitle = sandpeach!8,
    coltitle = black,
    boxrule = 0pt,
    frame hidden,
    borderline west = {0.6mm}{0mm}{sandpeach},
    fonttitle = \bfseries,
    before skip = 3ex,
    after skip = 0pt,
    sharp corners,
    rounded corners = northeast,
    breakable
}{problem}
\newtcbtheorem[]{solution}{}{
    enhanced,
    colback = seablue!7,
    coltitle = seablue!7,
    boxrule = 0pt,
    frame hidden,
    borderline west = {0.6mm}{0mm}{seablue},
    before skip = 0pt,
    after skip = 3ex,
    sharp corners,
    fonttitle = \tiny,
    rounded corners = southeast,
    attach title to upper = {},
    title after break = {},
    title = {},
    breakable
}{solution}

\newcommand{\legendre}[2]{\left(\frac{#1}{#2}\right)}
\newcommand{\floor}[1]{\left\lfloor#1\right\rfloor}

\begin{document}
% \begin{center}
%     Thanks to $\texttt{Albert Pavlovic}$ for the help!
%     \tcbincludegraphics[width=0.8\textwidth, colback=sandpeach!20, colframe=seablue!50!black, arc=4mm]{albuzz.jpg}
% \end{center}

\begin{problem}{}{}
    Carefully demonstrate that the distributive law holds for rational numbers.
\end{problem}
\begin{solution}{}{}
    Let $a,b,c,d,e,\text{ and }f\in\mathbb{Q}$.
    \begin{align*}
        \frac{a}{b}\left(\frac{c}{d}+\frac{e}{f}\right) &= \frac{a}{b}\left(\frac{cf+de}{df}\right) \\
        &= \frac{a(cf+de)}{bdf} \\
        &= \frac{acf+ade}{bdf} \\
        &= \frac{ac\bcancel{f}}{bd\bcancel{f}}+\frac{a\bcancel{d}e}{b\bcancel{d}f} \\
        \frac{a}{b}\left(\frac{c}{d}+\frac{e}{f}\right) &= \frac{ac}{bd}+\frac{ae}{bf}
    \end{align*}
    Thus, the distributive law holds for the rational numbers.
\end{solution}

\begin{problem}{}{}
    Given rational numbers $0<x<y$, show that $x^{-1}>y^{-1}$.
\end{problem}
\begin{solution}{}{}
    Since $y>x$, we can say that $y=x+\epsilon$ for some $\epsilon>0$. Thus:
    \begin{align*}
        \frac{\epsilon}{x(x+\epsilon)} &> 0 \quad\text{($x$ and $\epsilon$ are both positive)}\\
        \frac{\epsilon+x-x}{x(x+\epsilon)} &> 0 \\
        \frac{(x+\epsilon)-x}{x(x+\epsilon)} &> 0 \\
        \frac{1}{x}-\frac{1}{x+\epsilon} &> 0 \\
        \frac{1}{x}-\frac{1}{x+\epsilon}+\frac{1}{x+\epsilon} &> 0+\frac{1}{x+\epsilon} \\
        \frac{1}{x} &> \frac{1}{x+\epsilon} \\
        x^{-1} &> (x+\epsilon)^{-1} \\
        x^{-1} &> y^{-1}
    \end{align*}
    Thus, we have shown that $x^{-1}>y^{-1}$ when $0<x<y$.
\end{solution}

\begin{problem}{}{}
    Show that the rational numbers have the \textit{Archimedean property}: for any positive rational numbers $x$ and $y$, you can find a positive integer $n$ so that $nx>y$.
\end{problem}
\begin{solution}{}{}
    Let $a,b,c,d\in\mathbb{N}$, such that $a,b,c,d\geq1$, $x=\frac{a}{b}$, and $y=\frac{c}{d}$.
    \begin{align*}
        nx &> y \\
        n\cdot\frac{a}{b} &> \frac{c}{d} \\
        n\cdot\frac{a}{b}\cdot\frac{d}{c} &> \frac{c}{d}\cdot\frac{d}{c} \\
        n\cdot\frac{ad}{bc} &> 1 \\
    \end{align*}
    For simplicity, let $n=2bc$. Then:
    \begin{align*}
        2bc\cdot\frac{ad}{bc} &> 1 \\
        2ad &> 1 \\
        ad &> \frac{1}{2}
    \end{align*}
    Since $a,d\geq1$, $ad\geq1\cdot1\Rightarrow ad\geq1$. $1>\frac{1}{2}$, so $ad>\frac{1}{2}$ is true. Thus, the rational numbers have the Archimedean property.
\end{solution}
\end{document}