\documentclass{article}
\usepackage{enumitem}
\usepackage{fancyhdr}
\usepackage[a4paper, left=1in, right=1in, top=1in, bottom=1in]{geometry}
\usepackage{amsfonts}
\usepackage{amsmath}

\pagestyle{fancy}
\cfoot{}
\lhead{}
\chead{\Large\bf Number Theory Homework \# 3 - Divisibility}
\lhead{Sonit Sahoo}
\rhead{Page \thepage}
\setlength{\headheight}{10.0pt}

\begin{document}
\begin{enumerate}[label=\textbf{\arabic*}.]
    \Large
    \item
    \begin{enumerate}[label=\textbf{\alph*}.]
        \item $q=6,r=2$
        \item $q=12,r=0$
        \item $q=-2,r=10$
        \item $q=-3,r=7$
    \end{enumerate}
    \item 
    \begin{enumerate}[label=\textbf{\alph*}.]
        \item If we have an even integer $a$, then $2|a$ and we can say that $a=2k$ for some $k\in\mathbb{Z}$. Furthermore, we can say that any even integer will be in a similar form. Multiplying $a\cdot b$, where $a$ is an even integer and $b$ is some integer we get
        \[a\cdot b\]
        \[=2k\cdot b\]
        \[=2(kb)\]
        Since it is in this form, $ab$ is an even integer.
        \item If an integer $a$ is odd, it is not even, and thus $2\not{|}v$. Thus, it can be represented as $2k+1$ for some $k\in\mathbb{Z}$, since $2k$ is divisible by $2$, but $1$ isn't. We are looking for the product of two odd integers $x$ and $y$. Since they are odd, they can be written as $2m+1$ and $2n+1$ respectively for some $m,n\in\mathbb{Z}$.
        \[x\cdot y\]
        \[(2m+1)(2n+1)\]
        \[=2m(2n+1)+1(2n+1)\]
        \[=4mn+2m+2n+1\]
        \[=2(2mn+m+n)+1\]
        Since it is in this form, the product is odd. Thus, we have proven that the product of two odd integers is odd.
    \end{enumerate}
    \item We have proven that if $a|b$ and $c|d$ then $ab|cd$. So, if $a|b$ and $a|b$, then we can say $(a\cdot a)|(b \cdot b) \Rightarrow a^2|b^2$.
    \item Let the base case be $n=1$. $a|b$ as it's given, so the base case is true.
    For the inductive hypothesis, assume we have a $k\in\mathbb{Z}$ such that $a^k|b^k$. By the inductive hypothesis and since $a|b$, we can apply the theorem from problem 3 to get 
    \[(a^k\cdot a)|(b^k\cdot b)\]
    \[a^{k+1}|b^{k+1}\]
    This is the same formula for $k+1$; thus, by induction we have shown that $a^n|b^n$ for any positive integer $n$.
    \item If $n$ is composite, it has at least 2, not necessarily distinct, prime factors. Let the prime factors of $n$ be $d_1,\dots,d_k$. Thus, $n=d_1\times\dots\times d_k$. Now, let's assume that all the factors are $>\sqrt{n}$. Plugging this in we get $n>\sqrt{n}\times\dots\times\sqrt{n}$. Since, there $n$ has at least two prime factors, it is at minimum greater than $\sqrt{n}\cdot\sqrt{n}\Rightarrow n$. However, this is impossible as an integer $n$ can never be greater than itself. Thus, by contradiction, at least one of the prime factors of $n$ must be $\leq\sqrt{n}$.
    \item Let $p$ and $q$  be consecutive odd primes such that $p<q$. Since they are odd, they can be represented as $2k+1$ and $2j+1$ respectively for some $j,k\in\mathbb{Z}$.
    \[p+q\]
    \[=(2k+1)+(2j+1)\]
    \[=2k+2j+2\]
    \[=2(k+j+1)\]
    As we can see, $p+q$ is even, and thus $2|(p+q)$. From above, we defined $p<q$. Let's expand on this.
    \begin{align*}
        p&<q & p&<q\\
        p+p&<q+p & p+q&<q+q\\
        2p&<p+q & 2(k+j+1)&<2q\\
        2p&<2(k+j+1) & k+j+1&<q\\
        p&<k+j+1 \\
    \end{align*}
    So, since $p<q$, $p<k+j+1<q$. Since $p$ and $q$ are consecutive primes, this means that all the integers between them are not prime and, thus, have 1 factor or at least 2 factors. For it to have 1 factor, $k+j+1$ would have to be equal to 1 and thus $2(k+j+1)$ would have to equal to 2. However, this is not possible as there are no two positive consecutive primes that sum to 2. Thus, $k+j+1$ has to have at least 2 factors. Considering the additional factor of $2$ we discovered above, this means that $2(k+j+1)$, and thus $p+q$, has at least 3 prime factors.
\end{enumerate}
\end{document}