\documentclass{article}
\usepackage{enumitem}
\usepackage{fancyhdr}
\usepackage[a4paper, left=1in, right=1in, top=1in, bottom=1in]{geometry}
\usepackage{amsfonts}
\usepackage{amsmath}

\pagestyle{fancy}
\cfoot{}
\lhead{}
\chead{\Large\bf Intro to Proofs HW \# 5}
\lhead{Sonit Sahoo}
\rhead{Page \# \thepage}
\setlength{\headsep}{0.2in}

\begin{document}

\begin{enumerate}[label=\textbf{\arabic*}.]
    \item 
        \begin{enumerate}[label=\textbf{\alph*}.]
            \item If $a \equiv 3$ (mod 7), then
                \[a^2 \equiv 3^2 \text{ (mod 7)} \equiv 9 \text{ (mod 7)} \equiv 2 \text{ (mod 7)}\]
                \[5a \equiv 5\cdot3 \text{ (mod 7)} \equiv 15 \text{ (mod 7)} \equiv 1 \text{(mod 7)}\]
            Adding the two, we get
            \[a^2 + 5a \equiv (2+1) \text{ (mod 7)} \equiv 3 \text{ (mod 7)}\]
            Thus, it is \underline{true} that, $\forall a \in \mathbb{Z}$, if $a \equiv 3$ (mod 7), then ($a^2+5a) \equiv 3$ (mod 7).
            \item We can split modulo 7 into 7 equivalence classes for all $a \in \mathbb{Z}$, $[0],[1],[2],[3],[4],[5],[6],[7]$. Plugging all of these into $a^2+5a$ (mod 7), we get
            \[[0]: 0^2+5\cdot0 \text{ (mod 7)} \equiv 0 \text{ (mod 7)}\]
            \[[1]: 1^2+5\cdot1 \text{ (mod 7)} \equiv 6 \text{ (mod 7)}\]
            \[[2]: 2^2+5\cdot2 \text{ (mod 7)} \equiv 14 \text{ (mod 7)} \equiv 0 \text{ (mod 7)}\]
            \[[3]: 3^2+5\cdot3 \text{ (mod 7)} \equiv 24 \text{ (mod 7)} \equiv 3 \text{ (mod 7)}\]
            \[[4]: 4^2+5\cdot4 \text{ (mod 7)} \equiv 36 \text{ (mod 7)} \equiv 1 \text{ (mod 7)}\]
            \[[5]: 5^2+5\cdot5 \text{ (mod 7)} \equiv 50 \text{ (mod 7)} \equiv 1 \text{ (mod 7)}\]
            \[[6]: 6^2+5\cdot6 \text{ (mod 7)} \equiv 66 \text{ (mod 7)} \equiv 3 \text{ (mod 7)}\]
            From this, we see that $a^2+5a \equiv \text{ 3 (mod 7)}$ when $a \in [3] \lor [6]$. Thus, it is \underline{false} that if ($a^2+5a) \equiv 3$ (mod 7), then $a \equiv 3$ (mod 7) as $a \equiv 6$ (mod 7) is also possible.
        \end{enumerate}
    \item Let $a,b,c\in\mathbb{Z}$ and $n \in\mathbb{Z}$. Let $a\equiv a\text{ (mod n)}$.
    \[a\equiv a\text{ (mod n)}\]
    \[a-a\equiv a-a\text{ (mod n)}\]
    \[0\equiv0\text{ (mod n)}\]
    This is true, so $\sim$ is reflexive. Let $a\sim{b}$.
    \[a\equiv b\text{ (mod n)}\]
    \[a-a\equiv b-a\text{ (mod n)}\]
    \[-1\cdot0\equiv -1\cdot(b-a)\text{ (mod n)}\]
    \[0\equiv a-b\text{ (mod n)}\]
    \[b\equiv a\text{ (mod n)}\]
    Thus, $\sim$ is symmetric. Let $a\sim{b}$ and $b\sim{c}$.
    \[a\equiv b\text{ (mod n)}\]
    \[b\equiv c\text{ (mod n)}\]
    \[a+b\equiv b+c\text{ (mod n)}\]
    \[a+b-b\equiv b+c-b\text{ (mod n)}\]
    \[a\equiv c\text{ (mod n)}\]
    Thus, $a\sim{c}$ and $\sim$ is transitive. Since $\sim$ is reflexive, symmetric, and transitive, $\sim$ is an equivalence relation.
    \item Let $x\sim y$.
    \[|x|+|y|=10\]
    If we reorganize, we get
    \[|y|+|x|=10\]
    Thus, if $x\sim y$, then $y\sim x$, meaning that R is symmetric.\\\\
    If $R$ is reflexive, then $3\sim{3}$. Let us check this by plugging in:
    \[|x|+|y|=10\]
    \[|3|+|3|=10\]
    \[3+3=10\]
    \[6=10\]
    This means $3\not\sim{3}$, thus $R$ is not reflexive.\\\\
    Plugging in, we see that $4\sim{6}$ and $6\sim{-4}$  is true.
    \[|4|+|6|=10\]
    \[4+6=10\]
    \[10=10\]
    \[|6|+|-4|=10\]
    \[6+4=10\]
    \[10=10\]
    If $R$ is transitive, then $4\sim{-4}$. Let's check this.
    \[|4|+|-4|=10\]
    \[4+4=10\]
    \[8=10\]
    Obviously, $8\neq 10$ meaning $4\not\sim{-4}$, so $R$ is not transitive, by contradiction.
    \item 
        \begin{enumerate}[label=\textbf{\alph*}.]
                \item Let $a,b,c\in\mathbb{R}$ such that $\sin(a)=\sin(b)=\sin(c)$. Obviously, $\sin(a)=\sin(a)$, so $f(a)=f(a)$, and $a\sim{a}$, meaning $R$ is reflexive. Since $\sin(a)=\sin(b)$, $\sin(b)=\sin(a)$, so $f(a)=f(b)$ and $f(b)=f(a)$, thus $a\sim{b}$ and $b\sim{a}$, meaning $R$ is symmetric. Since $\sin(a)=\sin(b)=\sin(c)$, $\sin(a)=\sin(b)$, $\sin(b)=\sin(c)$, and $\sin(a)=\sin(c)$, so $a\sim{b}$, $b\sim{c}$, and $a\sim{c}$, meaning $R$ is transitive. Since $R$ is reflexive, symmetric, and transitive, it is an equivalence relation. 
            \item $[\pi]=\{k\pi|k\in\mathbb{Z}\}$ 
        \end{enumerate}
    \item 
        \begin{enumerate}[label=\textbf{\alph*}.]
            \item $[(3,4)]=\{c,d\in\mathbb{R}|c^2+d^2=25\}$
            \item The equivalence class $[(a,b)]$, where $a,b\in\mathbb{Z}$, can, generally, be represented as a circle whose radius is $\sqrt{a^2+b^2}$ and whose center is the origin.
        \end{enumerate}
\end{enumerate}

\end{document}