\documentclass{article}
\usepackage{enumitem}
\usepackage{fancyhdr}
\usepackage[a4paper, left=1in, right=1in, top=1in, bottom=1in]{geometry}
\usepackage{amsfonts}
\usepackage{amsmath}
\usepackage{tcolorbox}

\pagestyle{fancy}
\cfoot{}
\lhead{}
\chead{\Large\bf Number Theory HW \# 2 - Induction}
\lhead{Sonit Sahoo}
\rhead{Page \# \thepage}
\setlength{\headsep}{0.2in}

\begin{document}
\begin{enumerate}[label=\textbf{\arabic*}.]
    \Large
    \item Let the base case be $n=1$.
    \begin{align}
        \sum_{k=1}^{1}H_k&=H_1=\frac{1}{1}=1\\
        (1+1)H_1-1&=(2)\frac{1}{1}-1=2\cdot1-1=2-1=1
    \end{align}
    Since $1=1$, the base case holds.\\
    For the inductive step, assume we have a $j\in\mathbb{N}$, such that $j\geq1$ and $\sum_{k=1}^{j}H_k=(j+1)H_j-j$. Additionally, notice that $H_n+\frac{1}{n+1}=H_{n+1}$, so $H_n=H_{n+1}-\frac{1}{n+1}$.
    \begin{align*}
    \sum_{k=1}^{j+1}H_k &=(\sum_{k=1}^{j}H_k)+H_{j+1}\\
    &=(j+1)H_j-j+H_{j+1}\\
    &=(j+1)(H_{j+1}-\frac{1}{j+1})-j+H_{j+1}\\
    &=(j+1)H_{j+1}-(j+1)\frac{1}{j+1}-j+H_{j+1}\\
    &=(j+1)H_{j+1}+1\cdot H_{j+1}-1-j\\
    &=(j+1+1)H_{j+1}-1-j\\
    &=((j+1)+1)H_{j+1}-(j+1)
    \end{align*}
    This is the same formula for $k+1$; thus, by induction, we have shown that for any positive integer $n$, $\sum_{k=1}^{n}H_k=(n+1)H_n-n$.
    \item Let the base cases be $n=1$ and $n=2$. $f(1)=1$ and $f(2)=5$.
    \begin{align*}
        n&=1 & n&=2\\
        f(1)&=1 & f(2)&=5\\
        2^1+(-1)^1=2-1&=1 & 2^2+(-1)^2=4+1&=5
    \end{align*}
    Since $1=1$ and $5=5$, the base cases hold.\\
    For the inductive step, assume we have a $k\in\mathbb{N}$, such that $k\geq2$, $f(k)=2^k+(-1)^k$, and $f(k-1)=2^{k-1}+(-1)^{k-1}$. By definition,
    \begin{align*}
        f(k+1)&=f(k)+2f(k-1)\\
        &=2^k+(-1)^k+2(2^{k-1}+(-1)^{k-1})\\
        &=2^k+(-1)^k+2\cdot2^{k-1}+2\cdot(-1)^{k-1}\\
        &=2^k+(-1)^k+2^{k}+2\cdot(-1)^{k}\cdot(-1)\\
        &=2\cdot2^k+(-1)^k-2(-1)^{k}\\
        &=2^{k+1}-(-1)^{k}\\
        &=2^{k+1}+(-1)^1(-1)^{k}\\
        &=2^{k+1}+(-1)^{k+1}
    \end{align*}
    This is the same formula for $k+1$; thus, by induction, we have shown that for any positive integer $n$, $f(n)=2^n+(-1)^n$.
    %base case should be n=0 instead
    \item Let the base case be $n=1$. Obviously, $(x-y)|(x^1-y^1)$, as $(x-y)=1\cdot(x^1-y^1)$, so the base case holds.\\
    For the inductive step, assume we have a $k\in\mathbb{N}$, such that $k\geq1$ and $(x-y)|(x^k-y^k)$. Since $(x-y)|(x^k-y^k)$, $(x^k-y^k)=p\cdot(x-y)$, where $p$ is some polynomial.
    \begin{align*}
        x^{k+1}-y^{k+1}&=x^{k+1}-y^{k+1}+(-x^ky)+x^ky\\
        &=x^{k+1}+(-x^ky)+x^ky-y^{k+1}\\
        &=x^{k}\cdot x-x^k\cdot y+y\cdot x^k-y\cdot y^{k}\\
        &=x^{k}(x-y)+y(x^k-y^{k})\\
        &=x^{k}(x-y)+y\cdot p\cdot(x-y)\\
        &=(x-y)(x^{k}+y\cdot p)
    \end{align*}
    Since $(x-y)|((x-y)(x^{k}+y\cdot p))$, $(x-y)|(x^{k+1}-y^{k+1})$. Thus, by induction, we have shown that for any positive integer $n$, $(x-y)|(x^n-y^n)$.
    \item Let the base case be $n=1$. $1$ can be uniquely written as $2^0$. So, the base case holds.\\
    For the inductive step, we will use strong induction and assume that a $k\in\mathbb{N}$ and all $j\in\mathbb{N}$, such that $1\leq j\leq k$, can be uniquely written as a sum of distinct powers of $2$. Let $m\in\mathbb{N}$ such that $2^m$ is the greatest power of $2$ less than $k+1$, i.e. $2^m\leq k+1 <2^{m+1}$. Let $r\in\mathbb{N}$ such that $r=(k+1)-2^m$. Following this, $0\leq r<2^m$ and $r<k+1$. We can represent $k+1$ as $2^m+r$. There are 2 resulting cases, $r=0$ and $r>0$.\\
    \underbar{Case 1:} If $r=0$, then $k+1=2^m$. Thus, $k+1$ can be uniquely written as $2^m$.\\
    \underbar{Case 2:} If $r>0$, then recall that $r<k+1$, and thus $r\leq k$. By the strong inductive hypothesis, $r$ can be uniquely written as a sum of distinct powers of $2$. Moreover, because $r<2^m$ and $1+2+4+\cdots+2^{m-1}<2^{m}$, there will be no overlap between the powers used to represent $r$ and $2^m$. Therefore, $k+1$ can be uniquely written as $2^m+r$, a sum of distinct powers of 2.\\
    Thus, by strong induction, we have shown that every positive integer can be uniquely written as a sum of distinct powers of 2.
\end{enumerate}
\end{document}