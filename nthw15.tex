\documentclass{article}
\usepackage{enumitem}
\usepackage{fancyhdr}
\usepackage[a4paper, left=1in, right=1in, top=1in, bottom=1in]{geometry}
\usepackage{amsfonts}
\usepackage{amsmath}
\usepackage[many]{tcolorbox}
\usepackage{xcolor}
\usepackage{amssymb}
\usepackage{cancel}
\usepackage{siunitx}
\usepackage[mocha,styleAll]{catppuccinpalette}
\tcbuselibrary{skins}

\pagestyle{fancy}
\cfoot{}
\lhead{}

% 4769d8, ffc567
\definecolor{miamipurple}{RGB}{71, 105, 216}
\definecolor{miamiyellow}{RGB}{255, 197, 103}
\definecolor{miamipink}{RGB}{205, 126, 146}

\chead{\Large\bf NT HW \#15 - Continued Fractions I}
\lhead{Page \thepage}
\rhead{Sonit Sahoo}
\setlength{\headheight}{18.0pt}

\newcommand{\correct}{\textcolor{green}{\checkmark}}
\newcommand{\wrong}{\textcolor{red}{\times}}

\newtcbtheorem[]{problem}{Problem}{
    enhanced,
    colback = CtpSurface0,
    colbacktitle = CtpSurface0,
    coltitle = CtpRosewater,
    boxrule = 0pt,
    frame hidden,
    borderline west = {0.6mm}{0mm}{CtpBlue},
    fonttitle = \bfseries,
    before skip = 3ex,
    after skip = 0pt,
    sharp corners,
    rounded corners = northeast,
    breakable
}{problem}
\newtcbtheorem[]{solution}{}{
    enhanced,
    colback = CtpSurface0,
    coltitle = CtpSurface0,
    boxrule = 0pt,
    frame hidden,
    borderline west = {0.6mm}{0mm}{CtpPeach},
    before skip = 0pt,
    after skip = 3ex,
    sharp corners,
    fonttitle = \tiny,
    rounded corners = southeast,
    attach title to upper = {},
    title after break = {},
    title = {},
    breakable
}{solution}


\newcommand{\legendre}[2]{\left(\frac{#1}{#2}\right)}
\newcommand{\floor}[1]{\left\lfloor#1\right\rfloor}

\newcommand{\quotewrap}[1]{“#1”}

\begin{document}
% \begin{center}
%     Thanks to $\texttt{Albert Pavlovic}$ for the help!
%     \tcbincludegraphics[width=0.8\textwidth, colback=miamipurple!20, colframe=miamiyellow!50!black, arc=4mm]{albuzz.jpg}
% \end{center}

\begin{problem}{}{}
    \color{CtpText}
    Find the finite simple continued fraction [$a_0;a_1,\dots,a_n$] for the following rational numbers.
    \begin{enumerate}[label=\textbf{\arabic*}.]
        \item $\frac{4}{9}$
        \item $\frac{372}{1001}$
        \item $\frac{129}{51}$
    \end{enumerate}
\end{problem}
\begin{solution}{}{}
    \color{CtpText}
    \begin{enumerate}[label=\textbf{\arabic*}.]
        \item 
        \begin{align*}
            \frac{4}{9} &= 0 +\frac{1}{\frac{9}{4}} \\
            &= 0 + \frac{1}{2 + \frac{1}{\frac{4}{1}}} \\
            &= 0 + \frac{1}{2 + \frac{1}{4}} \\
            &= [0;2,4]
        \end{align*}
        \item 
        \begin{align*}
            \frac{372}{1001} &= 0 + \frac{1}{\frac{1001}{372}} \\
            &= 0 + \frac{1}{2 + \frac{372}{257}} \\
            &= 0 + \frac{1}{2 + \frac{1}{1+\frac{257}{115}}} \\
            &= 0 + \frac{1}{2 + \frac{1}{1+\frac{1}{2+\frac{115}{27}}}} \\
            &= 0 + \frac{1}{2 + \frac{1}{1+\frac{1}{2+\frac{1}{4+\frac{27}{7}}}}} \\
            &= 0 + \frac{1}{2 + \frac{1}{1+\frac{1}{2+\frac{1}{4+\frac{1}{3+\frac{7}{6}}}}}} \\
            &= 0 + \frac{1}{2 + \frac{1}{1+\frac{1}{2+\frac{1}{4+\frac{1}{3+\frac{1}{1+\frac{1}{6}}}}}}} \\
            &= [0;2,1,2,4,3,1,6]
        \end{align*}
        \item
        \begin{align*}
                \textcolor{CtpText}{\frac{129}{51}} &= \textcolor{CtpText}{2 + \frac{27}{51}} \\
                &= \textcolor{CtpText}{2 + \frac{1}{\frac{51}{27}}} \\
                &= \textcolor{CtpText}{2 + \frac{1}{1+\frac{24}{27}}} \\
                &= \textcolor{CtpText}{2 + \frac{1}{1+\frac{1}{1+\frac{3}{24}}}} \\
                &= \textcolor{CtpText}{2 + \frac{1}{1+\frac{1}{1+\frac{1}{\frac{24}{3}}}}} \\
                &= \textcolor{CtpText}{2 + \frac{1}{1+\frac{1}{1+\frac{1}{8+\frac{0}{3}}}}} \\
                &= \textcolor{CtpText}{2 + \frac{1}{1+\frac{1}{1+\frac{1}{8}}}} \\
                &= \textcolor{CtpText}{[2;1,1,8]}
            \end{align*}
    \end{enumerate}
\end{solution}
    
\begin{problem}{}{}
    \color{CtpText}
    For each finite simplified continued fraction, find the $p_k$, the $q_k$, and the convergents $C_k$.
    \begin{enumerate}[label=\textbf{\arabic*}.]
        \item $[2;1,5,7,3]$
        \item $[2;1,1,8]$
    \end{enumerate}
\end{problem}
\begin{solution}{}{}
    \color{CtpText}
    \begin{enumerate}[label=\textbf{\arabic*}.]
        \item
        \begin{align*}
        p_0&=2 & q_0&=1 & C_0&=\frac{2}{1}=2 \\
        p_1&=1\cdot2+1=3 & q_1&=1 & C_1&=\frac{3}{1}=3 \\
        p_2&=5\cdot3+2=17 & q_2&=5\cdot1+1=6 & C_2&=\frac{17}{6}\\
        p_3&=7\cdot17+3=122 & q_3&=7\cdot6+1=43 & C_3&=\frac{122}{43} \\
        p_4&=3\cdot122+17=383 & q_4&=3\cdot43+6=135 & C_4&=\frac{383}{135}
        \end{align*}
        \item
        \begin{align*}
            p_0&=2 & q_0&=1 & C_0&=\frac{2}{1}=2 \\
            p_1&=1\cdot2+1=3 & q_1&=1 & C_1&=\frac{3}{1}=3 \\
            p_2&=1\cdot3+2=5 & q_2&=1\cdot1+1=2 & C_2&=\frac{5}{2} \\
            p_3&=8\cdot5+3=43 & q_3&=8\cdot2+1=17 & C_3&=\frac{43}{17}
        \end{align*}
    \end{enumerate}
\end{solution}

\begin{problem}{}{}
    \color{CtpText}
    Explain why \underbar{2b} should have given \underbar{1c}, but did not.
\end{problem}
\begin{solution}{}{}
    \color{CtpText}
    Since the continued fractions are the same, we would have expected $C_3$ to be the same. However, this does not happen, because $\frac{129}{51}$ is not simplified. We see that in the final step that $\frac{3}{24}=\frac{1}{8+\frac{0}{3}}$. This last bit, $\frac{0}{3}$, \quotewrap{encodes} some information about the original rational, which is lost when we simplify it to 0. This simplification results in $C_3$ being a different rational.
\end{solution}

\begin{problem}{}{}
    \color{CtpText}
    Given a simple continued fraction $[a_0;a_1,\dots,a_n]$, prove the two relations:
    \begin{align}
        \frac{p_k}{p_{k-1}} &= [a_k;a_{k-1},\dots,a_0] \\
        \frac{q_k}{q_{k-1}} &= [a_k;a_{k-1},\dots,a_1]
    \end{align}
\end{problem}
\begin{solution}{}{}
    \color{CtpText}
    \textbf{Relation 1:}\\
    We can prove this with induction. The base case is $k=1$.
    \begin{align*}
        \frac{p_1}{p_0} &= \frac{a_1a_0 + 1}{a_0} \\
        &= a_1 + \frac{1}{a_0} \\
        &= [a_1;a_0]
    \end{align*}
    Thus, the base case holds. Now, let's assume that for $k=j$ such that $j\in\mathbb{N}$, the relation holds. Now for $k=j+1$:
    \begin{align*}
        \frac{p_{j+2}}{p_{j+1}} &= \frac{a_{j+2}p_{j+1} + p_j}{p_{j+1}} \\
        &= a_{j+2} + \frac{p_j}{p_{j+1}} \\
        &= a_{j+2} + \frac{1}{\frac{p_{j+1}}{p_j}} \\
        &= a_{j+2} + \frac{1}{[a_{j+1};a_j,\dots,a_0]} \\
        &= [a_{j+2};a_{j+1},\dots,a_0]
    \end{align*}
    Thus, since the inductive step holds, relation 1 holds for all $k\in\mathbb{N}$. \\
    \textbf{Relation 2:}\\
    We can prove this with induction. The base case is $k=1$.
    \begin{align*}
        \frac{q_1}{q_0} &= \frac{a_1}{1} \\
        &= a_1 \\
        &= [a_1]
    \end{align*}
    Thus, the base case holds. Now, let's assume that for $k=j$ such that $j\in\mathbb{N}$, the relation holds. Now for $k=j+1$:
    \begin{align*}
        \frac{q_{j+2}}{q_{j+1}} &= \frac{a_{j+2}q_{j+1} + q_j}{q_{j+1}} \\
        &= a_{j+2} + \frac{q_j}{q_{j+1}} \\
        &= a_{j+2} + \frac{1}{\frac{q_{j+1}}{q_j}} \\
        &= a_{j+2} + \frac{1}{[a_{j+1};a_j,\dots,a_1]} \\
        &= [a_{j+2};a_{j+1},\dots,a_1]
    \end{align*}
    Thus, since the inductive step holds, relation 2 holds for all $k\in\mathbb{N}$.
\end{solution}
\end{document}