\documentclass{article}
\usepackage{enumitem}
\usepackage{fancyhdr}
\usepackage[a4paper, left=1in, right=1in, top=1in, bottom=1in]{geometry}
\usepackage{amsfonts}
\usepackage{amsmath}
\usepackage[many]{tcolorbox}
\usepackage{xcolor}
\tcbuselibrary{skins}

\pagestyle{fancy}
\cfoot{}
\lhead{}
\chead{\Large\bf Number Theory HW \#7 - Modular Arithmetic}
\lhead{Page \thepage}
\rhead{Sonit Sahoo}
\setlength{\headheight}{18.0pt}
\definecolor{pastelblue}{rgb}{0.59, 0.83, 1}
\definecolor{pastelorange}{rgb}{1.0, 0.61, 0.84}
\definecolor{pastelgreen}{rgb}{0.84, 1, 0.62}

\newtcbtheorem[]{problem}{Problem}{
    enhanced,
    colback = pastelblue!5,
    colbacktitle = pastelblue!5,
    coltitle = black,
    boxrule = 0pt,
    frame hidden,
    borderline west = {0.6mm}{0mm}{pastelblue},
    fonttitle = \bfseries,
    before skip = 3ex,
    after skip = 0pt,
    sharp corners,
    rounded corners = northeast,
    breakable
}{problem}
\newtcbtheorem[]{solution}{}{
    enhanced,
    colback = pastelorange!5,
    coltitle = pastelorange!5,
    boxrule = 0pt,
    frame hidden,
    borderline west = {0.6mm}{0mm}{pastelorange},
    before skip = 0pt,
    after skip = 3ex,
    sharp corners,
    fonttitle = \tiny,
    rounded corners = southeast,
    attach title to upper = {},
    title after break = {},
    title = {},
    breakable
}{solution}
\newtcbtheorem[]{conjecture}{Conjecture}{
    enhanced,
    colback = pastelgreen!5,
    colbacktitle = pastelgreen!5,
    coltitle = black,
    boxrule = 0pt,
    frame hidden,
    borderline west = {0.6mm}{0mm}{pastelgreen},
    fonttitle = \bfseries,
    before skip = 3ex,
    after skip = 0pt,
    sharp corners,
    rounded corners = all,
    breakable
}{conjecture}

\begin{document}
\begin{problem}{}{}
    Show that if $a\equiv b\pmod{n}$ then $a\equiv b\pmod{k}$ for any $k$ which is a divisor of $n$.
\end{problem}
\begin{solution}{}{}
    Since $a\equiv b\pmod{n}$, $n|(a-b)$, and $a-b=nm$ for some $m\in\mathbb{Z}$. Since $k|n$, $n=ki$ for some $i\in\mathbb{Z}$. Therefore, $a-b=kim$, and $k|a-b$. Thus, $a\equiv b\pmod{k}$.
\end{solution}

\begin{problem}{}{}
    Show that the square of any odd number is congruent to 1, modulo 8.
\end{problem}
\begin{solution}{}{}
    Every odd number can be represented as $2k+1$ for some $k\in\mathbb{Z}$. Squaring gives us 
    \begin{align*}
        (2k+1)^2 &= 4k^2 + 4k + 1 \\
        &= 4k(k + 1) + 1
    \end{align*}
    If $k$ is even, then $8|4k$. If $k$ is odd, then $8|4(k+1)$. In either case, $8|(4k(k+1))$. Therefore, $(2k+1)^2\equiv 4k^2 + 4k + 1\equiv 4k(k + 1) + 1\equiv 0+1\equiv 1\pmod{8}$.
\end{solution}

\begin{problem}{}{}
    Given that $a^{10}\equiv74\pmod{650}$, find $(a,650)$.
\end{problem}
\begin{solution}{}{}
    $650=2\cdot5^2\cdot13$. From problem 1, we can create 3 equivalent congruences: $a^{10}\equiv74\equiv0\pmod{2}$, $a^{10}\equiv74\equiv4\pmod{5}$, and $a^{10}\equiv74\equiv9\pmod{13}$. The first congruence implies that $a$ is even and the other two imply that $5\not|a$ and $13\not|a$. Thus, $(a,650)=2$.
\end{solution}

\begin{problem}{}{}
    For $n=1,2,\dots,15$ calculate $(n-1)!\pmod{n}$. Do you state a pattern? State a conjecture.
\end{problem}
\begin{solution}{}{}
    \begin{center}
        \begin{tabular}{c|c}
            $n$ & $(n-1)!\pmod{n}$ \\
            \hline
            1 & 0 \\
            2 & 1 \\
            3 & 2 \\
            4 & 2 \\
            5 & 4 \\
            6 & 0 \\
            7 & 6 \\
            8 & 0 \\
            9 & 0 \\
            10 & 0 \\
            11 & 10 \\
            12 & 0 \\
            13 & 12 \\
            14 & 0 \\
            15 & 0 \\
        \end{tabular}
    \end{center}
    For all $n=1,2,\dots,15$, except $n=4$, $(n-1)\equiv-1\pmod{n}$ if $n$ is prime. If $n$ is composite, then $(n-1)!\equiv0\pmod{n}$.
    \begin{conjecture}{}{}
        For all $n\in\mathbb{N}$, if $n$ is prime, then $(n-1)!\equiv-1\pmod{n}$.
    \end{conjecture}
\end{solution}

\begin{problem}{}{}
    Find each reciprocal
    \begin{enumerate}[label=\textbf{\alph*}.]
        \item $7^{-1}\pmod{39}$
        \item $15^{-1}\pmod{111}$
        \item $12^{-1}\pmod{1331}$
    \end{enumerate}
\end{problem}
\begin{solution}{}{}
    \begin{enumerate}[label=\textbf{\alph*}.]
        \item $28\pmod{39}$
        \item $(15,111)=3$, so no inverse exists
        \item $111\pmod{1331}$
    \end{enumerate}
\end{solution}

\begin{problem}{}{}
    Solve each linear congruence:
    \begin{enumerate}[label=\textbf{\alph*}.]
        \item $7x\equiv22\pmod{39}$
        \item $15y\equiv86\pmod{111}$
        \item $15z\equiv87\pmod{111}$
        \item $12w\equiv1234\pmod{1331}$
    \end{enumerate}
\end{problem}
\begin{solution}{}{}
    \begin{enumerate}[label=\textbf{\alph*}.]
        \item As we found in 5a, the inverse of 7 modulo 39 is 28. Therefore, $x\equiv28\cdot22\equiv616\equiv31\pmod{39}$. 
        \item $(15,111)=3$, but $3\not|86$, so there are no solutions.
        \item $(15,87,111)=3$, so we'll divide by that: $5z\equiv29\pmod{37}$. The inverse of 5 modulo 37 is 15, so $z\equiv15\cdot29\equiv435\equiv28\pmod{37}$.
        \item $1234\equiv -97 \pmod{1331}$. As we found in 5c, the inverse of 12 modulo 1331 is 111. Therefore, $w\equiv111\cdot(-97)\equiv-10767\equiv1212\pmod{1331}$.
    \end{enumerate}
\end{solution}

\begin{problem}{}{}
    By casting out 9's and 11's find the missing digits a and b in the multiplication problem $38761 \times 29a37 = 11293b9257$.
\end{problem}
\begin{solution}{}{}
    We know modulus is multiplicative, so we can cast out 9's and 11's to find $a$ and $b$.
    \begin{align*}
        (3+8+7+6+1)\times(2+9+a+3+7) &\equiv (1+1+2+9+3+b+9+2+5+7) \pmod{9}\\
        7(3+a) &\equiv 3+b \pmod{9}\\
        21+7a &\equiv 3+b \pmod{9}\\
        7a-b &\equiv 0 \pmod{9}\\\\
        (3-8+7-6+1)\times(2-9+a-3+7) &\equiv (1-1+2-9+3-b+9-2+5-7) \pmod{11}\\
        8(8+a) &\equiv 1-b \pmod{11}\\
        64+8a &\equiv 1-b \pmod{11}\\
        7+8a+b &\equiv 0 \pmod{11}
    \end{align*}
    With these two congruences, we can attempt to find solutions for $a$ and $b$ knowing that $0\leq a,b\leq9$. Through some trial and error, we quickly find that $a=1$ and $b=7$.
\end{solution}

\begin{problem}{}{}
    Let $a^x\equiv a^y \equiv 1\pmod{n}$. Prove that $a^{(x,y)}\equiv1\pmod{n}$.
\end{problem}
\begin{solution}{}{}
    By Bezout's theorem, there exists a linear combination of $x$ and $y$ that equals $(x,y)$. Let $cx+dy=(x,y)$ for some $c,d\in\mathbb{Z}$. Since modulus is multiplicative we have that $a^{x\cdot c}\equiv a^{y\cdot d}\equiv 1\pmod{n}$. Therefore, $a^{(x,y)}\equiv a^{cx+dy}\equiv a^{cx}a^{dy}\equiv1\pmod{n}$.
\end{solution}

\begin{problem}{}{}
    Among real numbers $x^2=1$ if and only if $x=\pm1$. This is not always true in modular arithemtic. Show that it \textit{is} true for prime moduli: if $x^2\equiv1\pmod{p}$ where $p$ is prime, then $x\equiv\pm1\pmod{p}$.
\end{problem}
\begin{solution}{}{}
    Let's do some rearranging.
    \begin{align*}
        x^2 &\equiv 1 \pmod{p} \\
        x^2-1 &\equiv 0 \pmod{p} \\
        (x-1)(x+1) &\equiv 0 \pmod{p}
    \end{align*}
    Thus, there exists a $k\in\mathbb{Z}$, such that $(x-1)(x+1)=kp$. As we have previously shown, this indicates that $p|x-1$ or $p|x+1$. So, either $x+1\equiv 0 \pmod{p}\Rightarrow x\equiv -1 \pmod{p}$ or $x-1\equiv 0 \pmod{p}\Rightarrow x\equiv 1 \pmod{p}$. Thus, we have shown that if $x^2\equiv1\pmod{p}$ and $p$ is prime, then $x\equiv\pm1\pmod{p}$.
\end{solution}
\end{document}