\documentclass{article}
\usepackage{enumitem}
\usepackage{fancyhdr}
\usepackage[a4paper, left=1in, right=1in, top=1in, bottom=1in]{geometry}
\usepackage{amsfonts}
\usepackage{amsmath}
\usepackage[many]{tcolorbox}
\usepackage{xcolor}
% comic sans
\tcbuselibrary{skins}

\pagestyle{fancy}
\cfoot{}
\lhead{}
\chead{\Large\bf Number Theory Homework \# 5 - FTA \& GCD}
\lhead{Page \thepage}
\rhead{Sonit Sahoo}
\setlength{\headheight}{18.0pt}
\definecolor{pastelblue}{rgb}{0.68, 0.85, 0.90}
\definecolor{pastelorange}{rgb}{1.0, 0.7, 1.0}

\newtcbtheorem[]{problem}{Problem}{
    enhanced,
    colback = pastelblue!5,
    colbacktitle = pastelblue!5,
    coltitle = black,
    boxrule = 0pt,
    frame hidden,
    borderline west = {0.6mm}{0mm}{pastelblue!50},
    fonttitle = \bfseries,
    before skip = 3ex,
    after skip = 0pt,
    sharp corners,
    rounded corners = northeast,
    breakable
}{problem}
\newtcbtheorem[]{solution}{}{
    enhanced,
    colback = pastelorange!5,
    coltitle = pastelorange!5,
    boxrule = 0pt,
    frame hidden,
    borderline west = {0.6mm}{0mm}{pastelorange!50},
    before skip = 0pt,
    after skip = 3ex,
    sharp corners,
    fonttitle = \tiny,
    rounded corners = southeast,
    attach title to upper = {},
    title after break = {},
    title = {},
    breakable
}{solution}

\begin{document}
% \Large
\begin{problem}{}{}
    Let $H=\{1,5,9,13,\dots\}$ be the set of all positive integers in the form $4k+1$.
    \begin{enumerate}[label=\textbf{\alph*}.]
        \item Show that the set $H$ is closed under multiplication.
        \item List all numbers in $H$ that are less than 150 and \textit{not} Hilbert primes.
        \item Show that every number in $H$ has a factorization into Hilbert primes.
        \item Show that $H$ does \textbf{not} have (a) unique factorization by finding two different factorization(s) of 693 into Hilbert primes.
        \item What is the smallest number in $H$ with two different factorizations into Hilbert primes?
    \end{enumerate}
\end{problem}
\begin{solution}{}{}
    \begin{enumerate}[label=\textbf{\alph*}.]
        \item Let $a,b\in H$. Then $a=4k+1$ and $b=4j+1$ for some $k,l\in\mathbb{Z}$.
        \[a\cdot b\]
        \[=(4k+1)(4j+1)\]
        \[=16kj+4k+4j+1\]
        \[=4(4kj+k+j)+1\]
        This is the form of a Hilbert number, thus $H$ is closed under multiplication.
        \item Hilbert primes $<150$ are $1,5,13,\dots,149$ or $H_0,H_1,H_2,\dots H_{37}$. Since they are not primes, each $H_i$ is in the form $(4k+1)(4j+1)$ for some $k,j\in\mathbb{Z}$ such that $4k+1>1$ and $4j+1<H_i$. Let $k\leq j$. We can iterate through all possibilities of $k$ and $j$ to find non primes $<150$. Doing so, we get 9 such numbers: $25,45,65,81,85,105,117,125,145$.
        \item We will use strong induction. \\
        \underbar{Base Case:} $H_0=1$. Since 1 is a Hilbert prime, it factors into itself. Thus, the base case is true.\\
        \underbar{Inductive Hypothesis:} Assume that $H_k$ has a factorization into Hilbert primes for all $n$ and some $k$ such that $0\leq n<k$.
        There are two cases for $H_{k+1}$: either it is a Hilbert prime or it is not.\\
        \underbar{Case 1:} If $H_{k+1}$ is a Hilbert prime. Then $H_{k+1}=H_0\cdot H_{k+1}$. Thus, it has a factorization into Hilbert primes.\\
        \underbar{Case 2:} If $H_{k+1}$ is not a Hilbert prime, then $H_{k+1}=H_j\cdot H_i$ for some $j,i\in\mathbb{N}$, such that $H_j$ and $H_i$ are not 1 or $H_{k+1}$. By the inductive hypothesis, $H_j$ and $H_i$ have factorization into Hilbert primes. Thus, $H_{k+1}$ has a factorization into Hilbert primes.\\
        By strong induction, we have shown that for every $a\in H$, $H_a$ factorizes into Hilbert primes.
        \item $693=3^2\cdot 7\cdot 11$. From this we can see that $693=H_5\cdot H_8$ and $693=H_2\cdot H_{19}$. When we check, we see that $H_5,H_2,H_8,\text{ and }H_{19}<150$ and are not part of the list of non-Hilbert-primes that we found. So, they are Hilbert primes. Thus, $693$ has two different factorization into Hilbert primes, meaning that numbers in $H$ do not necessarily have unique factorization.
        \item Let there be $k\in\mathbb{N}$ such that $H_k$ has two different factorizations into Hilbert primes. Since we are looking for the smallest $k$, we can say $H_k=H_a\cdot H_b$ and $H_k=H_c\cdot H_d$ for some $a,b,c,d\in\mathbb{N}$ such that $H_a,H_b,H_c,\text{ and }H_d$ are Hilbert primes and distinct. Since $H_k=H_k$, $H_a\cdot H_b$ and $H_c\cdot H_d$ must have the same prime factorization. For them to be different Hilbert factorizations, we must arrange the factors differently. Obviously, $H_a$ or $H_b$ must be greater than 1 or there can't be two factorizations. Furthermore, at least one of $H_a$ and $H_b$ must be non-prime as it would factor to $H_0\cdot H_k=H_a\cdot H_b$ - not distinct. We can iterate through all the possibilities, setting 693 as a baseline since we already know that that has two Hilbert prime factorizations.
        \begin{center}
        \begin{tabular}{|c|c|c|c|c|}
        \hline
        $a$ & $b$ & $H_a$ & $H_b$ & $H_a \cdot H_b$ \\
        \hline
        1 & 2 & 5 & 9 & 45 \\
        1 & 6 & 5 & 25 & 125 \\
        1 & 8 & 5 & 33 & 165 \\
        \vdots & \vdots & \vdots & \vdots & \vdots \\
        2 & 12 & 9 & 49 & 441 \\
        \vdots & \vdots & \vdots & \vdots & \vdots \\
        5 & 5 & 21 & 21 & 441 \\
        \vdots & \vdots & \vdots & \vdots & \vdots \\
        \hline
        \end{tabular}
        \end{center}
        By doing such, we find that smallest number in $H$ with two different factorizations into Hilbert primes is 441.
    \end{enumerate}
\end{solution}

\begin{problem}{}{}
    Consider the polynomial $p(x)=x^7+2x^6+Ax^5+Bx^3+Cx^2-4x+1$ with all integer coefficients. Suppose you learn that $A+B+C=3$. Show that $p(x)$ cannot have rational roots.
\end{problem}
\begin{solution}{}{}
    By the rational root theorem, all roots $r$ of $p(x)$ must be in the form $\frac{s}{t}$ where $s|a_0$ and $t|a_n$. $a_0=a_n=1$. The only integers that divide 1 are 1 and $-1$. Thus, $\frac{s}{t}=\pm 1$. We are given $A+B+C=3$. If $p(1)$ or $p(-1)$ are roots, they will equal 1. Let us test both cases.
    \begin{align*}
        p(1)&=1^7+2(1)^6+A(1)^5+B(1)^3+C(1)^2-4(1)+1 \\
        &=1+2+A+B+C-4+1 \\
        &=A+B+C \\
    \end{align*}
    Obviously, $3 \neq 0$, so 1 is not a root.
    \begin{align*}
        p(-1)&=(-1)^7+2(-1)^6+A(-1)^5+B(-1)^3+C(-1)^2-4(-1)+1 \\
        &=-1+2-A-B+C+4+1 \\
        &=-A-B+C+6 \\
        -A-B+C&=6 \\
    \end{align*}
    Adding $A+B+C=3$ and $-A-B+C=6$ gives $2C=9\Rightarrow C=\frac{9}{2}$. However, this is not possible as $C$ is defined as an integer. Thus. $-1$ is not a root. Therefore, $p(x)$ does not have rational roots.
\end{solution}

\begin{problem}{}{}
    Compute (412, 600) and find $m$ and $n$ so that $412m + 600n = (412, 600)$.
\end{problem}
\begin{solution}{}{}
    We will utilize the extended Euclidean algorithm.
    \begin{center}
        \begin{tabular}{c c c c}
        $q$ & $r$ & $m$ & $n$ \\
        \hline
        & 600 & 0 & 1 \\
        1 & 412 & 1 & 0 \\
        2 & 188 & -1 & 1 \\
        5 & 36 & 3 & -2 \\
        4 & 8 & -16 & 11 \\
        2 & 4 & 67 & -46 \\
        \end{tabular}
    \end{center}
    Thus, $(412,600)=4$. $m=67-150t$ and $n=-46+103t$ for some $t\in\mathbb{Z}$.
\end{solution}

\begin{problem}{}{}
    Calculate $(840,595,476)$.
\end{problem}
\begin{solution}{}{}
    We will first compute $(840,595)$ to some $d$, then compute $(d,476)$ to get $(840,595,476)$.
    \begin{center}
        \begin{tabular}{c c}
        $q$ & $r$\\
        \hline
        & 840 \\
        1 & 595 \\
        2 & 245 \\
        2 & 105 \\
        2 & 35 \\
        3 & 0 \\
        \end{tabular}
    \end{center} 
    Thus, $(840,595)=35$. Now, we will compute $(35,476)$.
    \begin{center}
        \begin{tabular}{c c}
        $q$ & $r$\\
        \hline
        & 476 \\
        13 & 35 \\
        1 & 21 \\
        1 & 14 \\
        1 & 7 \\
        2 & 0 \\
        \end{tabular}
    \end{center}
    Thus, $(35,476)=7$. Therefore, $(840,595,476)=7$.
\end{solution}

\begin{problem}{}{}
    Search (on the web, in books, etc.) for an easy-to-state, easy-to-understand conjecture in number theory that has still not been proven. State the conjecture below.
\end{problem}
\begin{solution}{}{}
    The Erdős–Straus Conjecture, formulated in 1948, states that for all $n\in\mathbb{N},n\geq 2$, there exists $x,y,z\in\mathbb{N}$ such that
    \[\frac{4}{n}=\frac{1}{x}+\frac{1}{y}+\frac{1}{z}\]
    That is, the number $\frac{4}{n}$ can be expressed as the sum of three unit fractions. So far, it has been verified up to $n\leq 10^{17}$. Interestingly, it also seems to work when you allow negative unit fractions.
\end{solution}
\end{document}