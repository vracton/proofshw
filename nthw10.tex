\documentclass{article}
\usepackage{enumitem}
\usepackage{fancyhdr}
\usepackage[a4paper, left=1in, right=1in, top=1in, bottom=1in]{geometry}
\usepackage{amsfonts}
\usepackage{amsmath}
\usepackage[many]{tcolorbox}
\usepackage{xcolor}
\usepackage{cancel}
\usepackage{siunitx}
\tcbuselibrary{skins}

\pagestyle{fancy}
\cfoot{}
\lhead{}

\definecolor{chinayellow}{rgb}{0.82, 0.18, 0.6}
\definecolor{chinared}{rgb}{0.94, 0.82, 0.44}

\chead{\Large\bf Number Theory HW \#10 - Primality Testing}
\lhead{Page \thepage}
\rhead{Sonit Sahoo}
\setlength{\headheight}{18.0pt}

\newcommand\tagit{\addtocounter{equation}{1}\tag{\theequation}}

\newtcbtheorem[]{problem}{Problem}{
    enhanced,
    colback = chinayellow!7,
    colbacktitle = chinayellow!7,
    coltitle = black,
    boxrule = 0pt,
    frame hidden,
    borderline west = {0.6mm}{0mm}{chinayellow},
    fonttitle = \bfseries,
    before skip = 3ex,
    after skip = 0pt,
    sharp corners,
    rounded corners = northeast,
    breakable
}{problem}
\newtcbtheorem[]{solution}{}{
    enhanced,
    colback = chinared!7,
    coltitle = chinared!7,
    boxrule = 0pt,
    frame hidden,
    borderline west = {0.6mm}{0mm}{chinared},
    before skip = 0pt,
    after skip = 3ex,
    sharp corners,
    fonttitle = \tiny,
    rounded corners = southeast,
    attach title to upper = {},
    title after break = {},
    title = {},
    breakable
}{solution}

\begin{document}
\begin{problem}{}{}
    Show that 91 is a pseudoprime to bases $3$ and $17$.
\end{problem}
\begin{solution}{}{}
    We need to show that $3^{91}\equiv3\pmod{91}$ and $17^{91}\equiv17\pmod{91}$. Since $(3, 91)=(17, 91)=1$, we can use Euler's method. We need to find $\phi(91)$ first.
    \begin{align*}
        91 &= 7\times13 \\
        \phi(91) &= 91\left(1-\frac{1}{7}\right)\left(1-\frac{1}{13}\right) \\
        &= 91\left(\frac{6}{7}\right)\left(\frac{12}{13}\right) \\
        &= 72
    \end{align*}
    Thus, $3^{72}\equiv1\pmod{91}$ and $17^{72}\equiv1\pmod{91}$. Let's make a squaring table for $3$.
    \begin{align*}
        3^1 &\equiv 3\pmod{91} \\
        3^2 &\equiv 9\pmod{91} \\
        3^4 &\equiv 81\pmod{91} \\
        3^8 &\equiv 6561\equiv9\pmod{91} \\
        3^{16} &\equiv 81\pmod{91} \\
    \end{align*}
    \begin{align*}
        3^{91}&\equiv 3^{72+16+2+1}\pmod{91} \\
        &\equiv 3^{72}\cdot3^{16}\cdot3^2\cdot3^1\pmod{91} \\
        &\equiv 1\cdot81\cdot9\cdot3\pmod{91} \\
        &\equiv 2187\pmod{91} \\
        &\equiv 3\pmod{91}
    \end{align*}
    Now, let's make a squaring table for $17$.
    \begin{align*}
        17^1 &\equiv 17\pmod{91} \\
        17^2 &\equiv 289\equiv16\pmod{91} \\
        17^4 &\equiv 256\equiv74\pmod{91} \\
        17^8 &\equiv 5476\equiv16\pmod{91} \\
        17^{16} &\equiv 256\equiv74\pmod{91} \\
    \end{align*}
    \begin{align*}
        17^{91}&\equiv 17^{72+16+2+1}\pmod{91} \\
        &\equiv 17^{72}\cdot17^{16}\cdot17^2\cdot17^1\pmod{91} \\
        &\equiv 1\cdot74\cdot16\cdot17\pmod{91} \\
        &\equiv 20128\pmod{91} \\
        &\equiv 17\pmod{91}
    \end{align*}
    Thus, $91$ is a pseudoprime to bases $3$ and $17$.
\end{solution}

\begin{problem}{}{}
    Show that $2821=7\times13\times31$ is a Carmichael number.
\end{problem}
\begin{solution}{}{}
    We need to prove that for all $a\in\mathbb{Z}$ such that $(a,2821)=1$, $a^{2821}\equiv a\pmod{2821}$. $2821=7\times13\times31$. Since these are all prime, we can set up 3 congruences with Fermat's little theorem.
    \begin{align*}
        a^{6} &\equiv 1\pmod{7}\\
        a^{12} &\equiv 1\pmod{13}\\
        a^{30} &\equiv 1\pmod{31}
    \end{align*}
    \begin{align*}
        a^{2821} &\equiv a^{6\cdot471+1}\pmod{7}\\
        &\equiv a^{6\cdot471}\cdot a^{1}\pmod{7}\\
        &\equiv 1\cdot a\pmod{7}\\
        &\equiv a\pmod{7} \tagit\\
        a^{2821} &\equiv a^{12\cdot235+1}\pmod{13}\\
        &\equiv a^{12\cdot235}\cdot a^1\pmod{13}\\
        &\equiv 1\cdot a\pmod{13}\\
        &\equiv a\pmod{13} \tagit\\
        a^{2821} &\equiv a^{30\cdot94+1}\pmod{31}\\
        &\equiv a^{30\cdot94}\cdot a^{1}\pmod{31}\\
        &\equiv 1\cdot a\pmod{31}\\
        &\equiv a\pmod{31} \tagit
    \end{align*}
    Thus, we have our 3 final congruences:
    \begin{align*}
        a^{2821} &\equiv a\pmod{7}\\
        a^{2821} &\equiv a\pmod{13}\\
        a^{2821} &\equiv a\pmod{31}
    \end{align*}
    We could use the Chinese Remainder Theorem, but it is trivial to see that, since the bases are all the same and the moduli $|2821$, we can simply combine these congruences to get $a^{2821}\equiv a\pmod{2821}$. Thus, $2821$ is a Carmichael number.
\end{solution}
\setcounter{equation}{0}
\begin{problem}{}{}
    From the last homework, we know that 25 is a base-7 pseudoprime. Decide whether it is a strong pseudoprime.
\end{problem}
\begin{solution}{}{}
    From the last homework, we have $7^25\equiv7\pmod{25}$. $25-1=2^3\cdot3$, so we can use the Miller test. We can start with $7^3\pmod{25$} and continue square until we reach $-1$.
    \begin{align*}
        7^{3} &\equiv 343\pmod{25}\\
        &\equiv 325+18\pmod{25}\\
        &\equiv 18\pmod{25}\tagit\\
        7^{6} &\equiv 18^{2}\pmod{25}\\
        &\equiv 324\pmod{25}\\
        &\equiv 24\pmod{25}\\
        &\equiv -1\pmod{25}\tagit
    \end{align*}
    Since $7^6\equiv-1\pmod{25}$, it passes the test, but $25$ is not prime, so it is a strong pseudoprime to base $7$.
\end{solution}

\begin{problem}{}{}
    For each statement below, mark "Y" if the statement shows that $\num[group-separator={,}]{25326001}$ cannot be prime. Otherwise, answer "N".
    \begin{enumerate}[label=\textbf{(\alph*)}]
        \item $11251|25326001$
        \item $2^{25326001}\equiv2 \pmod{25326001}$
        \item $7^{25326001}\equiv5872860\pmod{25326001}$
        \item $3^{1582875}\equiv1\pmod{25326001}$
        \item $43^{1582875}\equiv12668627\pmod{25326001}$ and $43^{3165750}\equiv1\pmod{25326001}$
    \end{enumerate}
\end{problem}
\begin{solution}{}{}
    \begin{enumerate}[label=\textbf{(\alph*)}]
        \item $\underbar{Y}$, if there is factor that isn't 1 or $\num[group-separator={,}]{25326001}$, then it obviously can't be prime.
        \item $\underbar{N}$, while it seems to pass Fermat's little theorem, it could be a pseudoprime as the converse of a statement is not always true.
        \item $\underbar{Y}$, if it were prime, then $7^{25326001}\equiv7\pmod{25326001}$. Since it doesn't, it fails Fermat's little theorem and thus cannot be prime.
        \item $\underbar{N}$, $1582875$ is $25326001$ with all the 2s factored out and is thus the start of the Miller test. Since it is $\equiv1$ and, thus, passes the test, we can not definitively conclude anything about its primality.
        \item $\underbar{Y}$, $1582875$ is $25326001$ with all the 2s factored out and, thus, $43^{1582875}\equiv12668627\pmod{25326001}$ is the start of the Miller test. The next step is to square it, which is provided for us: $43^{3165750}\equiv1\pmod{25326001}$. We would continue squaring to see if we get a value $\equiv-1$, but since this second step is $\equiv1$, all future steps will also be $\equiv1$. Thus, since it is never $\equiv-1$, it fails the Miller test and cannot be prime.
    \end{enumerate}
\end{solution}
\end{document}