\documentclass{article}
\usepackage{enumitem}
\usepackage{fancyhdr}
\usepackage[a4paper, left=1in, right=1in, top=1in, bottom=1in]{geometry}
\usepackage{amsfonts}
\usepackage{amsmath}
\usepackage[many]{tcolorbox}
\usepackage{xcolor}
\tcbuselibrary{skins}

\pagestyle{fancy}
\cfoot{}
\lhead{}
\chead{\Large\bf Multi Variable Calculus Homework \# 4b}
\lhead{Page \thepage}
\rhead{Sonit Sahoo}
\setlength{\headheight}{18.0pt}
\definecolor{pastelblue}{rgb}{0.97, 0.97, 0.58}
\definecolor{pastelorange}{rgb}{0.596, 0.58, 0.97}

\newtcbtheorem[]{problem}{Problem}{
    enhanced,
    colback = pastelblue!5,
    colbacktitle = pastelblue!5,
    coltitle = black,
    boxrule = 0pt,
    frame hidden,
    borderline west = {0.6mm}{0mm}{pastelblue},
    fonttitle = \bfseries,
    before skip = 3ex,
    after skip = 0pt,
    sharp corners,
    rounded corners = northeast,
    breakable
}{problem}
\newtcbtheorem[]{solution}{}{
    enhanced,
    colback = pastelorange!5,
    coltitle = pastelorange!5,
    boxrule = 0pt,
    frame hidden,
    borderline west = {0.6mm}{0mm}{pastelorange},
    before skip = 0pt,
    after skip = 3ex,
    sharp corners,
    fonttitle = \tiny,
    rounded corners = southeast,
    attach title to upper = {},
    title after break = {},
    title = {},
    breakable
}{solution}

\begin{document}
\begin{problem}{}{}
    Let $g(x,y)=\sqrt[3]{xy}$.
    \begin{enumerate}[label=\textbf{\alph*}.]
        \item Is $g$ continuous at the origin?
        \item Calculate $\frac{\partial g}{\partial x}$ and $\frac{\partial g}{\partial y}$ when $xy\neq0$.
        \item Show that $g_x(0,0)$ and $g_y(0,0)$ exist.
        \item Is $g$ differentiable at the origin?
    \end{enumerate}
\end{problem}
\begin{solution}{}{}
    Solution
    \begin{enumerate}[label=\textbf{\alph*}.]
        \item We can check if $g$ is continuous at the origin by simply checking if the limit exists and is equal to $g(0,0)$.
        \[g(0,0)=\sqrt[3]{0\cdot0}=0\]
        \[\lim_{(x,y)\to(0,0)} g(x,y)=\sqrt[3]{0\cdot0}=0=g(0,0)\]
        Thus, $g$ is continuous at the origin.
        \item 
        \[\frac{\partial g}{\partial x}=\frac{1}{3}x^{-\frac{2}{3}}\sqrt[3]{y}=\frac{\sqrt[3]{y}}{3\sqrt[3]{x^2}}\]
        \[\frac{\partial g}{\partial y}=\frac{1}{3}y^{-\frac{2}{3}}\sqrt[3]{x}=\frac{\sqrt[3]{x}}{3\sqrt[3]{y^2}}\]
        \item Using the limit definition of partial derivatives,
        \begin{align*}
            g_x(0,0)&=\lim_{h\to0}\frac{g(0+h,0)-g(0,0)}{h}\\
            &=\lim_{h\to0}\frac{\sqrt[3]{h\cdot0}-0}{h}\\
            &=\lim_{h\to0}\frac{0}{h}\\
            &=0
        \end{align*}
        \begin{align*}
            g_y(0,0)&=\lim_{h\to0}\frac{g(0,0+h)-g(0,0)}{h}\\
            &=\lim_{h\to0}\frac{\sqrt[3]{0\cdot h}-0}{h}\\
            &=\lim_{h\to0}\frac{0}{h}\\
            &=0
        \end{align*}
        Thus, the partial derivatives exist.
        \item For $g$ to be differentiable at the origin, its partial derivatives must exist, and the following must hold:
        \[\lim_{(x,y)\to(0,0)}\frac{g(x,y)-L(x,y)}{||(x,y)-(0,0)||}=0\]
        where $L(x,y)=g(\vec{a})+\nabla g(\vec{a})\cdot ((x,y)-\vec{a})$, the linear approximation of $g$ at $\vec{a}$. We know the partial derivatives exist, so we only need to check the last requirement.
        \[L(0,0)=g(0,0)+g_x(0,0)(0-0)+g_y(0,0)(0-0)=0\]
        \begin{align*}
            \lim_{(x,y)\to(0,0)}\frac{g(x,y)-L(x,y)}{||(x,y)-(0,0)||}&=\lim_{(x,y)\to(0,0)}\frac{g(x,y)-0}{||(x,y)||}\\
            &=\lim_{(x,y)\to(0,0)}\frac{\sqrt[3]{xy}}{\sqrt{x^2+y^2}}\\
            &=\lim_{r\to0}\frac{\sqrt[3]{r\cos{\theta} \cdot r\sin{\theta}}}{\sqrt{r^2}}\\
            &=\lim_{r\to0}\frac{r^{\frac{2}{3}}\sqrt[3]{\cos{\theta} \cdot \sin{\theta}}}{|r|}\\
            &=\lim_{r\to0}|r|^{-\frac{1}{3}}\sqrt[3]{\cos{\theta} \cdot \sin{\theta}}\\
        \end{align*}
        There are two cases, $\cos{\theta}\cdot\sin{\theta}=0$ and $\cos{\theta}\cdot\sin{\theta}\neq0$. If $\cos{\theta}\cdot\sin{\theta}=0$, then the limit is 0. If $\cos{\theta}\cdot\sin{\theta}\neq0$, then the limit is not 0 as $|r|^{-\frac{1}{3}}$ goes to $\infty$ as $r\to 0$. Since this limit is not 0 from all directions, $g$ is not differentiable at the origin.
    \end{enumerate}
\end{solution}
\end{document}