\documentclass{article}
\usepackage{enumitem}
\usepackage{fancyhdr}
\usepackage[a4paper, left=1in, right=1in, top=1in, bottom=1in]{geometry}
\usepackage{amsfonts}
\usepackage{amsmath}
\usepackage{tcolorbox}
% comupter modern font

\pagestyle{fancy}
\cfoot{}
\lhead{}
\chead{\Large\bf Number Thoery HW \# 1 - Axioms}
\lhead{Sonit Sahoo}
\rhead{Page \# \thepage}
\setlength{\headsep}{0.2in}

% colors
\definecolor{pastelGreen}{HTML}{25b41f}
\definecolor{pastelRed}{HTML}{b41f1f}

\begin{document}
\begin{enumerate}[label=\textbf{\arabic*}.]
    \item 
    \begin{enumerate}[label=\textbf{\alph*}.]
        \item 
        \begin{tabular}{c|c}
        \textcolor{pastelGreen}{Passes} & \textcolor{pastelRed}{Fails} \\
        \hline 
        Addition & Addition \\\\
        \begin{minipage}{0.425\textwidth}
            \begin{enumerate}[label=-, itemsep=-3pt]
                \item Closure
                \item Compatibility
                \item Associativity
                \item Commutativity
            \end{enumerate}
        \end{minipage}
        &
        \begin{minipage}{0.425\textwidth}
            \begin{enumerate}[label=-, itemsep=-3pt]
                \item Identity - For there to be an identity, there must be a positive integer $b$ such that $b+a=a$ for all positive integers $a$. This means that $b$ must be greater than all $a$. This is obviously not possible as there is no greatest positive integer.
                \item Inverses - Since there is no identity, there cannot be an inverse. Furthermore, we do not have opposite numbers as all elements are positive integers.
            \end{enumerate}
        \end{minipage} \\\\
        Multiplication & Multiplication \\\\
        \begin{minipage}{0.425\textwidth}
            \begin{enumerate}[label=-, itemsep=-3pt]
                \item Closure
                \item Compatibility
                \item Associativity
                \item Commutativity
                \item Identity - $1 \times a$ will always equal $a$ because all positive integers are $\geq$ than 1.
                \item Distributivity over addition
            \end{enumerate}
        \end{minipage}
        &
        \begin{minipage}{0.425\textwidth}
            \begin{enumerate}[label=-, itemsep=-3pt]
                \item None
            \end{enumerate}
        \end{minipage}
        \end{tabular}
        \item
        \begin{tabular}{c|c}
        \textcolor{pastelGreen}{Passes} & \textcolor{pastelRed}{Fails} \\
        \hline 
        Addition & Addition \\\\
        \begin{minipage}{0.425\textwidth}
            \begin{enumerate}[label=-, itemsep=-3pt]
                \item Closure
                \item Compatibility
                \item Associativity
                \item Commutativity
                \item Identity - $(0,0)$
                \item Inverses - If an ordered pair $(a,b)$ exists, it has an invers $(-(a,b))$ or $(-a,-b)$.
            \end{enumerate}
        \end{minipage}
        &
        \begin{minipage}{0.425\textwidth}
            \begin{enumerate}[label=-, itemsep=-3pt]
                \item None
            \end{enumerate}
        \end{minipage} \\\\
        Multiplication & Multiplication \\\\
        \begin{minipage}{0.425\textwidth}
            \begin{enumerate}[label=-, itemsep=-3pt]
                \item Closure
                \item Compatibility
                \item Associativity
                \item Commutativity
                \item Identity - $(1,1)$
                \item Distributivity over addition
            \end{enumerate}
        \end{minipage}
        &
        \begin{minipage}{0.425\textwidth}
            \begin{enumerate}[label=-, itemsep=-3pt]
                \item None
            \end{enumerate}
        \end{minipage}
        \end{tabular}
    \end{enumerate}
    \item 
        By the reflexivity of equals $0=0$. By compatibility of addition with equality, we can add $(x+y)+(-(x+y))+x+(-x)+y+(-y)$ to both sides.
        \[0=0\]
        \[0+(x+y)+(-(x+y))+x+(-x)+y+(-y)=0+(x+y)+(-(x+y))+x+(-x)+y+(-y)\]
        By the additive inverse
        \[0+(x+y)+(-(x+y))+0+0=0+0+x+(-x)+y+(-y)\]
        \[(x+y)+(-(x+y))=+x+(-x)+y+(-y)\]
        By compatibility, we can add $(-x)+(-y)$ to both sides.
        \[(x+y)+(-(x+y))+(-x)+(-y)=x+(-x)+y+(-y)+(-x)+(-y)\]
        \[x+y+(-(x+y))+(-x)+(-y)=x+(-x)+y+(-y))+(-x)+(-y)\]
        By much associativity
        \[x+(-x)+y+(-y)+(-(x+y))=x+(-x)+(-x)+y+(-y)+(-y)\]
        Thus, by the additive inverse
        \[0+0+(-(x+y))=0+(-x)+0+(-y)\]
        \[(-(x+y))=(-x)+(-y)\]
        Thus, we have shown that $(-(x+y))=(-x)+(-y)$.
    \item By left distributivity
        \[z \cdot (x+y)=z \cdot x + z \cdot y\]
        Let $w=(x+y)$.
        \[z \cdot w=z \cdot x + z \cdot y\]
        By the commutativity of multiplication, $z \cdot w = w \cdot z$, so
        \[w \cdot z=z \cdot x + z \cdot y\]
        Substituting $(x+y)$ back in for $w$,
        \[(x+y) \cdot z=z \cdot x + z \cdot y\]
        Thus, we have proven right distributivity.
    \item 
    \begin{enumerate}[label=\textbf{\alph*}.]
        \item In class, we found
        \[0 \cdot a=0\]
        By the additive inverse, $1+(-1)=0$, thus
        \[(1+(-1)) \cdot a=0\]
        By right distributivity,
        \[1 \cdot a+(-1) \cdot a=0\]
        By the multiplicative identity, $1 \cdot a=a$, so
        \[a+(-1) \cdot a=0\]
        By additive compatibility,
        \[a+(-1) \cdot a+(-a)=0+(-a)\]
        By additive associativity,
        \[a+(-a)+(-1) \cdot a=(-a)\]
        By the additive inverse,
        \[0+(-1) \cdot a=(-a)\]
        \[(-1) \cdot a=(-a)\]
        Thus, we have shown that $(-1) \cdot a=(-a)$.
        \item Let us start with $(-a) \cdot b$. From part \underbar{a}, we know $a=(-1)\cdot a$. Thus,
        \[(-a) \cdot b=((-1)\cdot a) \cdot b\]
        By multiplicative associativity,
        \[=(-1)\cdot (a \cdot b)\]
        By part \underbar{a} again, we can say that this equals $(-(a \cdot b))$. Therefore, we have shown that $(-a) \cdot b=-(a \cdot b)$.
    \end{enumerate}
\end{enumerate}
\end{document}
