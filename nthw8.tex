\documentclass{article}
\usepackage{enumitem}
\usepackage{fancyhdr}
\usepackage[a4paper, left=1in, right=1in, top=1in, bottom=1in]{geometry}
\usepackage{amsfonts}
\usepackage{amsmath}
\usepackage[many]{tcolorbox}
\usepackage{xcolor}
\usepackage{gradient-text}
\tcbuselibrary{skins}

\pagestyle{fancy}
\cfoot{}
\lhead{}

\definecolor{chinayellow}{rgb}{1, 0.87, 0}
\definecolor{chinared}{rgb}{0.87, 0.16, 0.06}

\chead{\Large\bf \#8 - \gradientRGB{Chinese Remainder Theorem}{255,222,0}{222,41,14}}
\lhead{Page \thepage}
\rhead{Sonit Sahoo}
\setlength{\headheight}{18.0pt}

\newcommand\tagit{\addtocounter{equation}{1}\tag{\theequation}}

\newtcbtheorem[]{problem}{}{
    enhanced,
    colback = chinayellow!5,
    colbacktitle = chinayellow!5,
    coltitle = black,
    boxrule = 0pt,
    frame hidden,
    borderline west = {0.6mm}{0mm}{chinayellow},
    fonttitle = \bfseries,
    before skip = 3ex,
    after skip = 0pt,
    sharp corners,
    rounded corners = northeast,
    breakable
}{problem}
\newtcbtheorem[]{solution}{}{
    enhanced,
    colback = chinared!5,
    coltitle = chinared!5,
    boxrule = 0pt,
    frame hidden,
    borderline west = {0.6mm}{0mm}{chinared},
    before skip = 0pt,
    after skip = 3ex,
    sharp corners,
    fonttitle = \tiny,
    rounded corners = southeast,
    attach title to upper = {},
    title after break = {},
    title = {},
    breakable
}{solution}

\begin{document}
\begin{problem}{}{}
    Solve the system of linear congruences:
    \begin{align*}
        x &\equiv 4 \pmod{11} \\
        x &\equiv 7 \pmod{12} \\
        x &\equiv -3 \pmod{13}
    \end{align*}
\end{problem}
\begin{solution}{}{}
    $N=11 \cdot 12 \cdot 13 = 1716$. So, we must calculate $m_n$ of the following
    \begin{align*}
        \frac{N}{11} = 156m_1 \equiv 2m_1 \equiv 1 \pmod{11}\\
        \frac{N}{12} = 143m_2 \equiv 11m_2 \equiv 1 \pmod{12} \\
        \frac{N}{13} = 132m_3 \equiv 2m_3 \equiv 1 \pmod{13}
    \end{align*}
    \begin{center}
        \begin{tabular}{c c c c}
            $q$ & $r$ & $m_1$ & $y$ \\
            \hline
            & 11 & 0 & 1 \\
            5 & 2 & 1 & 0 \\
            1 & 1 & -5 & 1 \\
        \end{tabular}
        \quad
        \begin{tabular}{c c c c}
            $q$ & $r$ & $m_2$ & $y$ \\
            \hline
            & 12 & 0 & 1 \\
            1 & 11 & 1 & 0 \\
            1 & 1 & $-1$ & 1
        \end{tabular}
        \quad
        \begin{tabular}{c c c c}
            $q$ & $r$ & $m_3$ & $y$ \\
            \hline
            & 13 & 0 & 1 \\
            6 & 2 & 1 & 0 \\
            1 & 1 & $-6$ & 1
        \end{tabular}
    \end{center}
    Thus, we have $m_1=6$, $m_2=11$, and $m_3=7$. Now we can calculate the solution:
    \begin{align*}
        x &\equiv 4 \cdot 156 \cdot 6 + 7 \cdot 143 \cdot 11 + (-3) \cdot 132 \cdot 7 \pmod{1716} \\
        &\equiv 3744 + 11011 - 2772 \pmod{1716} \\
        &\equiv 11983 \pmod{1716} \\
        &\equiv 1687 \pmod{1716} \\
    \end{align*}
\end{solution}

\begin{problem}{}{}
    Solve the system of linear congruences:
    \begin{align*}
        2x &\equiv 1 \pmod{3} \\
        3x &\equiv 2 \pmod{5} \\
        4x &\equiv 3 \pmod{7} \\
        5x &\equiv 4 \pmod{11}
    \end{align*}
\end{problem}
\begin{solution}{}{}
    $N=3\cdot5\cdot7\cdot11=1155$. Now, let's find the inverses of the given congruences. 
    \begin{center}
        \begin{tabular}{c c c c}
            $q$ & $r$ & $m_1$ & $y$ \\
            \hline
            & 3 & 0 & 1 \\
            1 & 2 & 1 & 0 \\
            1 & 1 & -1 & 1
        \end{tabular}
        \quad
        \begin{tabular}{c c c c}
            $q$ & $r$ & $m_2$ & $y$ \\
            \hline
            & 5 & 0 & 1 \\
            1 & 3 & 1 & 0 \\
            2 & 1 & -1 & 1 \\
        \end{tabular}
        \quad
        \begin{tabular}{c c c c}
            $q$ & $r$ & $m_3$ & $y$ \\
            \hline
            & 7 & 0 & 1 \\
            1 & 4 & 1 & 0 \\
            3 & 1 & -1 & 1
        \end{tabular}
        \quad
        \begin{tabular}{c c c c}
            $q$ & $r$ & $m_4$ & $y$ \\
            \hline
            & 11 & 0 & 1 \\
            2 & 5 & 1 & 0 \\
            1 & 1 & -2 & 1
        \end{tabular}
        \end{center}
    Thus, we found 2, 4, 6, and 9 respectively. Let's plug these in
    \begin{align*}
        x &\equiv 1\cdot2 \equiv 2 \pmod{3} \\
        x &\equiv 2\cdot4 \equiv 3 \pmod{5} \\
        x &\equiv 3\cdot6 \equiv 4 \pmod{7} \\
        x &\equiv 4\cdot9 \equiv 36 \equiv 3 \pmod{11}
    \end{align*}
    Now, we must solve for $m_n$ in
    \begin{align*}
        \frac{N}{3} = 385m_1 \equiv m_1 \equiv 1 \pmod{3}\\
        \frac{N}{5} = 231m_2 \equiv m_2 \equiv 1 \pmod{5} \\
        \frac{N}{7} = 165m_3 \equiv 4m_3 \equiv 1 \pmod{7} \\
        \frac{N}{11} = 105m_4 \equiv 6m_4 \equiv 1 \pmod{11}
    \end{align*}
    \begin{center}
        \begin{tabular}{c c c c}
            $q$ & $r$ & $m_1$ & $y$ \\
            \hline
            & 3 & 0 & 1 \\
            1 & 2 & 1 & 0 \\
            1 & 1 & -1 & 1 \\
            & & 1 & -1 \\
        \end{tabular}
        \quad
        \begin{tabular}{c c c c}
            $q$ & $r$ & $m_2$ & $y$ \\
            \hline
            & 5 & 0 & 1 \\
            1 & 3 & 1 & 0 \\
            1 & 2 & -1 & 1 \\
            & & 1 & -1 \\
        \end{tabular}
        \quad
        \begin{tabular}{c c c c}
            $q$ & $r$ & $m_3$ & $y$ \\
            \hline
            & 7 & 0 & 1 \\
            1 & 4 & 1 & 0 \\
            1 & 3 & -1 & 1 \\
            1 & 1 & 2 & -1 \\
        \end{tabular}
        \quad
        \begin{tabular}{c c c c}
            $q$ & $r$ & $m_4$ & $y$ \\
            \hline
            & 11 & 0 & 1 \\
            2 & 5 & 1 & 0 \\
            1 & 1 & -2 & 1 \\
            & & 2 & -2 \\
        \end{tabular}
    \end{center}
    Thus, we have $m_1=1$, $m_2=1$, $m_3=2$, and $m_4=2$. Now we can calculate the solution:
    \begin{align*}
        x &\equiv 2\cdot385\cdot1 + 3\cdot231\cdot1 + 4\cdot165\cdot2 + 3\cdot105\cdot2 \pmod{1155} \\
        &\equiv 770 + 693 + 1320 + 630 \pmod{1155} \\
        &\equiv 3413 \pmod{1155} \\
        &\equiv 839 \pmod{1155} \\
    \end{align*}
\end{solution}

\begin{problem}{}{}
    Solve the system of linear congruences:
    \begin{align*}
        x &\equiv 2 \pmod{2} \\
        x &\equiv 4 \pmod{6} \\
        x &\equiv 2 \pmod{14} \\
        x &\equiv 10 \pmod{15}
    \end{align*}
\end{problem}
\begin{solution}{}{}
    We see that for any pair of congruences in the system, the GCD of the moduli divides the difference of the residues. Thus, a solution exists. Let's start from the first congruence working our way down.
    \begin{align*}
    x&\equiv 2\pmod{2} \tagit \\
    x&=2+2k_1 \text{ for } k_1\in\mathbb{Z} \\
    2+2k_1 &\equiv 4 \pmod{6} \tagit \\
    2k_1 &\equiv 2 \pmod{6} \\
    k_1 &\equiv 1 \pmod{3}\\
    k_1 &= 1+3k_2 \text{ for } k_2\in\mathbb{Z}\\
    x&=2+2(1+3k_2) \\
    x &= 4+6k_2 \\
    4+6k_2 &\equiv 2 \pmod{14} \tagit \\
    6k_2 &\equiv -2 \pmod{14} \\
    3k_2 &\equiv 6 \pmod{7} \\
    3k_2 &= 6+7k_3 \text{ for } k_3\in\mathbb{Z}\\
    x&=4+2(6+7k_3) \\
    x &= 16+14k_3 \\
    16+14k_3 &\equiv 10 \pmod{15} \tagit \\
    14k_3 &\equiv -6 \pmod{15} \\
    14k_3 &\equiv 9 \pmod{15} \\
    -k_3 &\equiv 9 \pmod{15} \\
    k_3 &\equiv 6 \pmod{15} \\
    k_3 &= 6+15k_4 \text{ for } k_4\in\mathbb{Z}\\
    x&=16+14(6+15k_4) \\
    x &= 100+210k_4 \\
    x &\equiv 100 \pmod{210} \tagit
    \end{align*}
\end{solution}

\begin{problem}{}{}
    Prove that there are arbitrarily long strings of consecutive integers in which every one of the numbers is divisible by a square (other than $1!$).
\end{problem}
\begin{solution}{}{}
    First, let $P=\{p_1,p_2,\dots,p_n\}$ be the set of all prime numbers where all $p_i$ are unique. We have previously proven that there are infinite prime numbers, so $|P|=\infty$. Now, we want to find a string of $n$ consecutive integers starting with $x$. Let's say that $x$ is divisible by $p_1^2$, $x+1$ is divisible by $p_{1+1}^2=p_{2}^2$, $\dots$, and $x+n-1$ is divisible by $p_{n}^2$. Now, let's rewrite in terms of congruences.
    \begin{align*}
        x \equiv 0 &\pmod{p_1^2} \\
        x+1 \equiv 0 &\pmod{p_2^2} \\
        &\vdots \\
        x+n-1 \equiv 0 &\pmod{p_n^2}
    \end{align*}
    Rearranging gives us:
    \begin{align*}
        x \equiv 0 &\pmod{p_1^2} \\
        x \equiv -1 &\pmod{p_2^2} \\
        &\vdots \\
        x \equiv -n+1 &\pmod{p_n^2}
    \end{align*}
    Since $p_1,p_2,\dots p_n$ are all unique primes, any pair of congruences has coprime moduli. Thus, we can use the Chinese Remainder Theorem. It tells us that there exists a solution to this system of congruences. Since we can choose $n$ to be arbitrarily large, we can construct arbitrarily long sequences of consecutive integers, each divisible by a square.
\end{solution}

\begin{problem}{}{}
    Find the least positive residue of $3^{984} \pmod{31}$.
\end{problem}
\begin{solution}{}{}
    Luckily, $31$ is prime, so we can simply use Fermat's Little Theorem. From it, we know $3^{31-1} = 3^{30} \equiv 1 \pmod{31}$.
    \begin{align*}
        3^{984} &\equiv 3^{30 \cdot 32 + 24} \pmod{31} \\
        &\equiv (3^{30})^{32} \cdot 3^{24} \pmod{31} \\
        &\equiv 1^{32} \cdot 3^{24} \pmod{31} \\
        &\equiv 3^{24} \pmod{31}
    \end{align*}
    Now, let's compute $3^{24} \pmod{31}$ with successive squares.
    \begin{align*}
        3^2 &\equiv 9 \pmod{31} \\
        3^4 &\equiv 81 \equiv 19 \pmod{31} \\
        3^8 &\equiv 361 \equiv 20 \pmod{31} \\
        3^{16} &\equiv 400 \equiv 28 \pmod{31}
    \end{align*}
    \begin{align*}
        3^{24} &\equiv 3^{16+8} \pmod{31} \\
        &\equiv 3^{16} \cdot 3^8 \pmod{31} \\
        &\equiv 28 \cdot 20 \pmod{31} \\
        &\equiv 560 \pmod{31} \\
        &\equiv 2 \pmod{31}
    \end{align*}
    Thus, the least possible residue of $3^{984} \pmod{31}$ is $2$.
\end{solution}

\begin{problem}{}{}
    Find the least positive residue of $3^{984} \pmod{360}$.
\end{problem}
\begin{solution}{}{}
    $360=2^3\cdot3^2\cot5$. We can then split this into congruences of $8$, $9$, and $5$ and combine with the Chinese Remainder Theorem. Let's calculate the congruences.
    \begin{align*}
        3^{984}&=9^{492}\\
        &\equiv 1^{492} \pmod{8} \\
        &\equiv 1 \pmod{8}
    \end{align*}
    \begin{align*}
        3^{984}&=9^{492}\\
        &\equiv 0^{492} \pmod{9} \\
        &\equiv 0 \pmod{9}
    \end{align*}
    By Fermat's Little Theorem, we can say $3^{5-1}=3^4\equiv 1 \pmod{5}$.
    \begin{align*}
        3^{984}&=3^{4\cdot246}\\
        &= (3^4)^{246}\\
        &\equiv 1^{246} \pmod{5}\\
        &\equiv 1 \pmod{5}
    \end{align*}
    Thus, we have 3 congruences with coprime moduli.
    \begin{align*}
        x &\equiv 1 \pmod{8} \\
        x &\equiv 0 \pmod{9} \\
        x &\equiv 1 \pmod{5}
    \end{align*}
    Now, we can solve this using the Chinese remainder Theorem. $N=8\cdot9\cdot5=360$. Then, we need to find all inverses $m_n$.
    \begin{align*}
        \frac{N}{8} = 45m_1 \equiv 5m_1 \equiv 1 \pmod{8}\\
        \frac{N}{9} = 40m_2 \equiv 4m_2 \equiv 1 \pmod{9} \\
        \frac{N}{5} = 72m_3 \equiv 2m_3 \equiv 1 \pmod{5}
    \end{align*}
    \begin{center}
        \begin{tabular}{c c c c}
            $q$ & $r$ & $m_1$ & $y$ \\
            \hline
            & 8 & 0 & 1 \\
            5 & 5 & 1 & 0 \\
            1 & 3 & -1 & 1 \\
            1 & 2 & 2 & -1 \\
            1 & 1 & -3 & 2 \\
        \end{tabular}
        \quad
        \begin{tabular}{c c c c}
            $q$ & $r$ & $m_2$ & $y$ \\
            \hline
            & 9 & 0 & 1 \\
            4 & 4 & 1 & 0 \\
            2 & 1 & -2 & 1 \\
        \end{tabular}
        \quad
        \begin{tabular}{c c c c}
            $q$ & $r$ & $m_3$ & $y$ \\
            \hline
            & 5 & 0 & 1 \\
            2 & 2 & 1 & 0 \\
            2 & 1 & -2 & 1 \\
        \end{tabular}
    \end{center}
    Thus, we have $m_1=5$, $m_2=7$, and $m_3=3$. Now we can calculate the solution.
    \begin{align*}
        x &\equiv 1\cdot45\cdot5 + 0\cdot40\cdot7 + 1\cdot72\cdot3 \pmod{360} \\
        &\equiv 225 + 0 + 216 \pmod{360} \\
        &\equiv 441 \pmod{360} \\
        &\equiv 81 \pmod{360} \\
    \end{align*}
    Thus, the least possible residue of $3^{984} \pmod{360}$ is $81$.
\end{solution}

\end{document}