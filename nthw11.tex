\documentclass{article}
\usepackage{enumitem}
\usepackage{fancyhdr}
\usepackage[a4paper, left=1in, right=1in, top=1in, bottom=1in]{geometry}
\usepackage{amsfonts}
\usepackage{amsmath}
\usepackage[many]{tcolorbox}
\usepackage{xcolor}
\usepackage{cancel}
\usepackage{siunitx}
\tcbuselibrary{skins}

\pagestyle{fancy}
\cfoot{}
\lhead{}

\definecolor{chinayellow}{rgb}{0.82, 0.18, 0.6}
\definecolor{chinared}{rgb}{0.94, 0.82, 0.44}

\chead{\Large\bf Number Theory HW \#11 - Legendre Symbols}
\lhead{Page \thepage}
\rhead{Sonit Sahoo}
\setlength{\headheight}{18.0pt}

\usepackage{amssymb,xcolor}
\newcommand{\correct}{\textcolor{green}{\checkmark}}
\newcommand{\wrong}{\textcolor{red}{\times}}

\newtcbtheorem[]{problem}{Problem}{
    enhanced,
    colback = chinayellow!7,
    colbacktitle = chinayellow!7,
    coltitle = black,
    boxrule = 0pt,
    frame hidden,
    borderline west = {0.6mm}{0mm}{chinayellow},
    fonttitle = \bfseries,
    before skip = 3ex,
    after skip = 0pt,
    sharp corners,
    rounded corners = northeast,
    breakable
}{problem}
\newtcbtheorem[]{solution}{}{
    enhanced,
    colback = chinared!7,
    coltitle = chinared!7,
    boxrule = 0pt,
    frame hidden,
    borderline west = {0.6mm}{0mm}{chinared},
    before skip = 0pt,
    after skip = 3ex,
    sharp corners,
    fonttitle = \tiny,
    rounded corners = southeast,
    attach title to upper = {},
    title after break = {},
    title = {},
    breakable
}{solution}

\newcommand{\legendre}[2]{\left(\frac{#1}{#2}\right)}

\begin{document}
% \begin{center}
%     Thanks to $\texttt{Albert Pavlovic}$ for the help!
%     \tcbincludegraphics[width=0.8\textwidth, colback=chinayellow!20, colframe=chinared!50!black, arc=4mm]{albuzz.jpg}
% \end{center}

\begin{problem}{}{}
    Showing all work, compute $\left(\frac{10}{19}\right)$ using:
    \begin{enumerate}[label=\textbf{(\alph*)}]
        \item Euler's Criterion
        \item Gauss' Lemma
    \end{enumerate}
\end{problem}
\begin{solution}{}{}
    \begin{enumerate}[label=\textbf{(\alph*)}]
        \item 
        \begin{align*}
            10^{\frac{19-1}{2}}&\equiv10^9\pmod{19}\\
            &\equiv (10^2)^4\cdot10\pmod{19}\\
            &\equiv 100^4\cdot10\pmod{19}\\
            &\equiv 5^4\cdot10\pmod{19}\\
            &\equiv 25^2\cdot10\pmod{19}\\
            &\equiv 6^2\cdot10\pmod{19}\\
            &\equiv 36\cdot10\pmod{19}\\
            &\equiv 17\cdot10\pmod{19}\\
            &\equiv 170\pmod{19}\\
            &\equiv 18\pmod{19}\\
            &\equiv -1\pmod{19}
        \end{align*}
        \item 
        \[\frac{19-1}{2}=9\]
        \[10,20,30,40,50,60,70,80,90\]
        \[10,1,11,2,12,3,13,4,14 \text{ (take (mod 11))}\]
        $5$ of these are greater than $\frac{19}{2}$, so $\left(\frac{10}{19}\right)=(-1)^5=-1$.
    \end{enumerate}
\end{solution}

\begin{problem}{}{}
    Let $p$ be an odd prime, and $(a,p)=1$. Show that
    \[\left(\frac{a}{p}\right)+\left(\frac{2a}{p}\right)+\left(\frac{3a}{p}\right)+\dots+\left(\frac{(p-1)a}{p}\right)=0\]
\end{problem}
\begin{solution}{}{}
    First, let there be a quadratic residue $a$ such that $x^2\equiv a\pmod{p}$ where $x$ is a least nonnegative residue. Notice that if we input $-x$, $(-x)^2\equiv x^2\equiv a\pmod{p}$. So $-x$ gives the same quadratic residue $a$ as $x$. So, for any quadratic residue $a$, there are two least residues $x$ and $-x$ such that $x^2\equiv a\pmod{p}$. Thus, since there are $p-1$ residues, the number of quadratic residues is $\frac{p-1}{2}$. Now, since legendre symbols are (sort of) multiplicative, we can the given equation as 
    \[=\legendre{a}{p}\left(\legendre{1}{p}+\legendre{2}{p}+\dots+\legendre{p-2}{p}+\legendre{p-1}{p}\right)\]
    As we found above, half of these are quadratic residues, and, consequently half are not quadratic residues. None of them $=0$. Thus, half are $=1$ and half are $=-1$. So the total sum of the inner legendre symbols are $0$. Thus, we have:
    \begin{align*}
        &=\legendre{a}{p}\cdot0\\
        &=0
    \end{align*}
\end{solution}

\begin{problem}{}{}
    Determine the values $c$ for which $3x^2+5x+c\equiv0\pmod{17}$ can be solved.
\end{problem}
\begin{solution}{}{}
    This quadratic has a solution if the discriminant is a quadratic residue. The discriminant is given by $5^2-4\cdot3\cdot c=25-12c$. We need to find $c$ such that $25-12c\equiv r\pmod{17}$. Let's first find the quadratic residues modulo $17$:
    \begin{align*}
        0^2&\equiv0\pmod{17}\\
        1^2&\equiv1\pmod{17}\\
        2^2&\equiv4\pmod{17}\\
        3^2&\equiv9\pmod{17}\\
        4^2&\equiv16\pmod{17}\\
        5^2&\equiv25\equiv8\pmod{17}\\
        6^2&\equiv36\equiv2\pmod{17}\\
        7^2&\equiv49\equiv15\pmod{17}\\
        8^2&\equiv64\equiv13\pmod{17}\\
    \end{align*}
    $a^2\equiv(a-17)^2\pmod{17}$, so the other possible squares will not give any more unique quadratic residues. Thus, we have quadratic residues $0,1,2,4,8,9,13,15,16$. We can now find $c$ such that $25-12c\equiv r\pmod{17}$.
    \begin{align*}
        25-12c&\equiv r\pmod{17}\\
        -12c&\equiv r-25\pmod{17}\\
        5c&\equiv r+9\pmod{17}\\
    \end{align*}
    The inverse of $5$ modulo $17$ is $7$, so we can multiply both sides by $7$ to get:
    \begin{align*}
        5c&\equiv r+9\pmod{17}\\
        c&\equiv 7(r+9)\pmod{17}\\
        c&\equiv 7r+63\pmod{17}\\
        c&\equiv 7r+12\pmod{17}\\
    \end{align*}
    Now we can plug in the quadratic residues to find $c$:
    \begin{align*}
        r&=0\implies c\equiv12\pmod{17}\\
        r&=1\implies c\equiv19\equiv2\pmod{17}\\
        r&=2\implies c\equiv26\equiv9\pmod{17}\\
        r&=4\implies c\equiv40\equiv6\pmod{17}\\
        r&=8\implies c\equiv68\equiv0\pmod{17}\\
        r&=9\implies c\equiv75\equiv7\pmod{17}\\
        r&=13\implies c\equiv103\equiv1\pmod{17}\\
        r&=15\implies c\equiv117\equiv15\pmod{17}\\
        r&=16\implies c\equiv124\equiv5\pmod{17}
    \end{align*}
    Thus, we have $c=\{0,1,2,5,6,7,9,12,15\}$.
\end{solution}

\begin{problem}{}{}
    Compute $\left(\frac{999}{2027}\right)$.
\end{problem}
\begin{solution}{}{}
    \begin{align*}
        \legendre{999}{2027}&=\legendre{3}{2027}^3\cdot\legendre{37}{2027} \hspace{0.3in}(999=3^3\cdot37)\\
        &=(\legendre{2027}{3}\cdot(-1)^{\frac{3-1}{2}\frac{2027-1}{2}})^3\cdot\legendre{2027}{37} \hspace{0.3in}(2027\not\equiv3\pmod{4})\\
        &=(\legendre{2}{3}\cdot-1)^3\cdot\legendre{2027}{37}\\
        &=(-1\cdot-1)^3\cdot\legendre{29}{37} \hspace{0.3in}(2^{\frac{3-1}{2}}\equiv2\equiv-1\pmod{3})\\
        &=1\cdot\legendre{37}{29} \hspace{0.3in}(37\not\equiv3\pmod{4})\\
        &=1\cdot\legendre{8}{29}\\
        &=1\cdot\legendre{2}{29}^3\\
        &=1\cdot((-1)^{\frac{29^2-1}{8}})^3 \hspace{0.3in}(\text{By theorem proven in class})\\
        &=1\cdot((-1)^{\frac{840}{8}})^3\\
        &=1\cdot((-1)^{105})^3\\
        &=1\cdot(-1)^3\\
        &=1\cdot-1\\
        &=-1
    \end{align*}
\end{solution}
\end{document}