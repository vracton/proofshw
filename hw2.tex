\documentclass{article}
\usepackage{enumitem}
\usepackage{fancyhdr}
\usepackage[a4paper, left=1in, right=1in, top=1in, bottom=1in]{geometry}
\usepackage{amsfonts}
\usepackage{amsmath}

\pagestyle{fancy}
\cfoot{}
\lhead{}
\chead{\Large\bf Intro to Proofs HW \# 2}
\lhead{Sonit Sahoo}
\rhead{Page \# \thepage}
\setlength{\headsep}{0.2in}

\begin{document}

\begin{enumerate}[label=\textbf{\arabic*}.]
    \item 
    \begin{enumerate}[label=\textbf{\alph*}.]
        \item $\forall{x}\exists{y} \ P(x,y)$
        \item $\exists{x}\exists{y} \ \neg{P(x,y)}$
        \item $\exists!{x} \ P(x,Alaska)$
        \item $\exists{x}\forall{y} \ (\neg(P(x,y) \oplus y=Illinois)$
    \end{enumerate}
    \item 
    \begin{enumerate}[label=\textbf{\alph*}.]
        \item $\neg (\forall x \forall y \ (x+y=y+x)) \Rightarrow \exists x \exists y \ (x+y \neq y+x)$
        \item $\neg (\forall x \exists y \ ((x \neq y) \wedge (x+y=3)) \Rightarrow \exists x \forall y \ ((x = y) \vee (x+y \neq 3)$
    \end{enumerate}
    \item 
    Let $x \in \mathbb{Z}$.
    \[(x+1)^2-x^2\]
    \[x^2+2x+1-x^2\]
    \[2x+1\]
    Remember that every odd number can be written in the form $2k+1$ where $k \in \mathbb{Z}$. Since the difference of the squares of some $x$ and some $x+1$ can be written in this form, it is odd. Therefore, since $x$ can be substituted for any integer, we can get any and all odd integers from the difference of two squares.
    \item Let $x_1,x_2,...,x_n \in \mathbb{R}$. We can prove with contradiction by assuming the opposite where all $x_n$ are less than the average.
    \[x_n < \frac{\sum_{k=1}^{n}x_n}{n}\]
    We can then add all the terms of $x_n$ together. This gives us
    \[x_1+x_2+...+x_n<\frac{\sum_{k=1}^{n}x_n}{n}+\frac{\sum_{k=1}^{n}x_n}{n}+...+\frac{\sum_{k=1}^{n}x_n}{n}\]
    \[\sum_{k=1}^{n}x_n<\sum_{k=1}^{n}x_n\]
    Obviously, a number can not be less than itself, proving that, by contradiction, at least one of the real numbers $x_1,x_2,...,x_n$ is greater than or equal to the average of all $x_n$.
    \item 
    \begin{enumerate}[label=\textbf{\alph*}.]
        \item Let $a,n,j,k \in \mathbb{Z}$. Since $a|(8n+7)$ and $a|(4n+1)$, we can rewrite in the form: 
        \begin{align}\notag
            8n+7 &= aj   &   4n+1 &= ak \\ \notag
            \frac{aj-7}{8} &= n &  \frac{ak-1}{4} &= n
        \end{align}
        \[\frac{aj-7}{8} = \frac{ak-1}{4}\]
        \[\frac{aj-7}{8}*8 = \frac{ak-1}{4}*8\]
        \[aj-7=2ak-2\]
        \[aj-2ak=5\]
        \[a(j-2k)=5\]
        Since $j-2k$ is an integer, this means that $a$ divides $5$.
        \item Let $a,n,j,k \in \mathbb{Z}$. Since $a|(9n+5)$ and $a|(6n+1)$, we can rewrite in the form: 
        \begin{align}\notag
            9n+5 &= aj   &   6n+1 &= ak \\ \notag
            \frac{aj-5}{9} &= n &  \frac{ak-1}{6} &= n
        \end{align}
        \[\frac{aj-5}{9} = \frac{ak-1}{6}\]
        \[\frac{aj-5}{9}*18 = \frac{ak-1}{6}*18\]
        \[2aj-10=3ak-3\]fefoekoe
        \[2aj-3ak=7\]
        \[a(2j-3k)=7\]
        Since $2j-3k$ is an integer, this means that $a$ divides $7$.
        \item Let $n \in \mathbb{Z}$ and be odd. This means it can be represented in the form $2k+1$ where $k \in \mathbb{Z}$. Let us compute $n^4+4n^2+11$ and simplify.
        \[n^4+4n^2+11\]
        \[(2k+1)^4+4(2k+1)^2+11\]
        \[16k^4+32k^3+24k^2+8k+1+16k^2+16k+4+11\]
        \[16k^4+32k^3+40k^2+24k+16\]
        \[8(2k^4+4k^3+5k^2+3k+2)\]
        Since $2k^4+4k^3+5k^2+3k+2$ is an integer, we have proved that $n^4+4n^2+11$ is equal to $8$ times some integer when $n$ is odd, meaning that $8|(n^4+4n^2+11)$.
        \item We can prove this false by letting $n=1$ substituting:
        \[8|(n^4+n^2+2n)\]
        \[8|(1^4+1^2+2*1) \Rightarrow 8|(1+1+2) \Rightarrow 8|4\]
        Obviously, 4 is not divisible by 8 meaning that the proposition is false.
    \end{enumerate}
    \item We will prove by contradiction. Let $m,n,p \in \mathbb{Z}$ and let $p$ be odd. The equation $x^2+2mx+2p=0$, which we assume has integer solutions, can be written as
    \[x^2+2mx=-2p\]
    We can then do casework on the value of $x$ depending on whether it is even or odd. \\\\
    Case 1: Let $n$ be an even integer and let us substitute for $x$. Since $n$ is even, it will be in the form $2k$ such that $k \in \mathbb{Z}$.
    \[(2k)^2+2m(2k)=-2p\]
    \[4k^2+4mk=-2p\]
    \[-2k^2-2mk=p\]
    \[-2(k^2+mk)\]
    $p$ is in the form $2k$, meaning that $p$ is even in this case. This contradicts the assumption that $p$ is odd.\\\\
    Case 2: Let $n$ be an odd integer and let us substitute for $x$. Since it is odd, it will be in the form $2k+1$ such that $k \in \mathbb{Z}$.
    \[(2k+1)^2+2m(2k+1)=-2p\]
    \[4k^2+4k+1+4mk+2m=-2p\]
    \[-2k^2-2k-\frac{1}{2}-2mk-m=p\]
    Since $-2k^2-2k-2mk-m \in \mathbb{Z}$ and it is subtracted by $\frac{1}{2}$, $p$ is not an integer, which contradicts the assumption that $p$ is an integer. \\\\
    As we can see in the above cases, when the solution is an even integer, $p$ must be an even integer, and when the solution is an odd integer, $p$ can not be an integer. This proves that, by contradiction, if $p$ is odd, then there is no integer solution of $x^2+2mx+2p=0$.
\end{enumerate}

\end{document}
