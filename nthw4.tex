\documentclass{article}
\usepackage{enumitem}
\usepackage{fancyhdr}
\usepackage[a4paper, left=1in, right=1in, top=1in, bottom=1in]{geometry}
\usepackage{amsfonts}
\usepackage{amsmath}
\usepackage[many]{tcolorbox}
\usepackage{xcolor}
\tcbuselibrary{skins}

\pagestyle{fancy}
\cfoot{}
\lhead{}
\chead{\Large\bf Number Theory Homework \# 4 - GCD}
\lhead{Page \thepage}
\rhead{Sonit Sahoo}
\setlength{\headheight}{18.0pt}
\definecolor{pastelblue}{rgb}{0.68, 0.85, 0.90}
\definecolor{pastelorange}{rgb}{1.0, 0.85, 0.7}

\newtcbtheorem[]{problem}{Problem}{
    enhanced,
    colback = pastelblue!5,
    colbacktitle = pastelblue!5,
    coltitle = black,
    boxrule = 0pt,
    frame hidden,
    borderline west = {0.6mm}{0mm}{pastelblue},
    fonttitle = \bfseries,
    before skip = 3ex,
    after skip = 0pt,
    sharp corners,
    rounded corners = northeast,
    breakable
}{problem}
\newtcbtheorem[]{solution}{}{
    enhanced,
    colback = pastelorange!5,
    coltitle = pastelorange!5,
    boxrule = 0pt,
    frame hidden,
    borderline west = {0.6mm}{0mm}{pastelorange},
    before skip = 0pt,
    after skip = 3ex,
    sharp corners,
    fonttitle = \tiny,
    rounded corners = southeast,
    attach title to upper = {},
    title after break = {},
    title = {},
    breakable
}{solution}

\begin{document}
\Large
\begin{problem}{}{}
    Recall that if $d = (a, b)$ with $a = de$ and $b = df$, then $(e, f) = 1$. That is, if we factor out the GCD from two numbers, the remaining numbers are relatively prime. Show that it is necessary to divide \textit{both} numbers by the GCD. That is, find numbers $a$ and $b$ with $d=(a,b)$, and $a=de$, such that $(e,b)\neq1$.
\end{problem}
\begin{solution}{}{}
    Let $a=4 \text{ and } b=6$. 
    \[d=(a,b)=(4,6)=2\]
    \[a=de \Rightarrow 2=4e \Rightarrow e=2\]
    \[(e,b)=(2,6)=2\neq1\]
    Thus, both numbers must be divided by the GCD of $a$ and $b$.
\end{solution}

\begin{problem}{}{}
    Which positive integers have exactly three distinct positive divisors? Four?
\end{problem}
\begin{solution}{}{}
    For any given $a\in\mathbb{N}$, $a$ has at least two factors: $1$ and $a$. If we want additional factors, then there must exist a prime factor $f$ such that $1<f<a$. Since $f$ is a factor, $f|a \Rightarrow a=fk$ for some $k\in\mathbb{N},k>1$. \\
    
    For $a$ to have only 3 factors, $k$ must be one of $1,f,\text{ or } a$. $f\cdot a$ is obviously too big and $f\cdot 1=f$ which is less than $a$, so $k$ must be $f$. Thus, for a number to have exactly 3 factors, it must be in the form $f^2$ where $f$ is a prime integer. \\

    For $a$ to have only 4 factors, $k$ must not be $1,f,\text{ or } a$. Thus, there are two cases, $k$ is prime or $k$ is not prime. If $k$ is prime, then the factors of $a$ are $1,f,k, \text{ and } a$. So $a$ will be in the form $fk$ where $f$ and $k$ are prime integers. If $k$ is not prime, then it must be a product of some number of primes. However, if $k$ is a product of primes that are not already factors of $a$, then $a$ will have more than 4 factors. So, $k$ must be factors of a prime which divides $a$. The only such prime is $f$, so $k$ is a power of $f$. We see that, in order to have 4 factors, $k$ must be $f^2$, as the factors of $a$ will then be $1,f,f^2, \text{ and } a$. Thus, $a$ must in the form $f^3$ where $f$ is a prime integer. So, for $a$ to have exactly 4 factors, it must be either in the form $fk$ or $f^3$ where $f$ and $k$ are prime integers.
\end{solution}

\begin{problem}{}{}
    Prove that if $(a, b) = 1$ then $(a, b^n) = 1$ for any positive integer $n$. Then go on to show that $(a^m, b^n) = 1$ for any positive integers $m$ and $n$.
\end{problem}
\begin{solution}{}{}
    We will use strong induction to prove this.\\
    \underline{Base Case:} Let $n=1$. $(a,b^1)=(a,b)=1 \text{ since given.}$ Thus, our base case is true.\\
    \underline{Inductive Hypothesis:} Assume that for a $k \text{ and all } j\in\mathbb{N}$ such that $1\leq k\leq j$, $(a, b^k) = 1$. \\
    We need to show that $(a, b^{k+1}) = 1$. 
    \[(a, b^{k+1})\]
    \[=(a, b^k\cdot b)\]
    By the inductive hypothesis, $(a, b^k) = 1$. So
    \[=(a, b)\]
    By the inductive hypothesis once more, $(a,b)=1$. Thus, $(a, b^{k+1}) = 1$.\\
    Thus, by induction, we have shown that $(a, b^n) = 1$ for $n\in\mathbb{N}$. By letting $b^n$ be some arbitrary integer $c$, we can repeat the same induction to prove $(a^m,c)=1$ for all positive integers $m$. We can then substitute $c$ for $b^n$ to show that $(a^m, b^n) = 1$ for all positive integers $m$ and $n$.
\end{solution}

\begin{problem}{}{}
    Prove the converse to \#3. That is, if there are positive integers $m$ and $n$ such that $(a^m, b^n) = 1$ then $a$ and $b$ are relatively prime.    
\end{problem}
\begin{solution}{}{}
    Let us instead prove the contrapositive. Let $d=(a,b)$ and $d>1$. $d|a$ and $d|b$. Following this, we can say $d|(a\cdot a^{m-1})\Rightarrow d|a^m$ and $d|(b\cdot b^{n-1})\Rightarrow d|b^n$. Since $d$ is a common divisor, $(a^m,b^n)\geq d\neq 1$. Since we have proven the contrapositive, the original statement is true: for any $m,n\in\mathbb{Z}$, if $(a^m,b^n)=1$ then $(a,b)=1$.
\end{solution}

\begin{problem}{}{}
    Prove this corollary: $(a^n, b^n) = (a, b)^n$ (even when a and b are not relatively prime).
\end{problem}
\begin{solution}{}{}
    Let $d=(a,b)$. Then, there exists a $e,f\in\mathbb{Z}$ such that $a=de$ and $b=df$ and $(e,f)=1$.
    \[(a^n, b^n)\]
    \[=((de)^n,(df)^n)\]
    \[=(d^ne^n,d^nf^n)\]
    Since $(e,f)=1$ and following our proof from problem 3, we can say that $(d^ne^n,d^nf^n) = d^n\cdot(e^n,f^n)=d^n\cdot 1=d^n$.
    Since $(a,b)=d$, $(a,b)^n=d^n$. Since $d^n=d^n$, we have shown that $(a^n, b^n) = (a, b)^n$.
\end{solution}

\begin{problem}{(Extra Credit)}{}
    Given that $(a, b) = 1$, what can you determine (with proof, of course!) about $(a^2 + b^2, a + b)$?
\end{problem}
\begin{solution}{}{}
    We can rewrite $(a^2 + b^2, a + b)$:
    \[(a^2 + b^2, a + b)\]
    \[=((a + b)^2-2ab, a + b)\]
    \[=((a + b)^2-2ab+(a+b)((-1)(a+b)), a + b)\]
    \[=(-2ab, a + b)\]
    \[=(2ab, a + b)\]
    Since $(a,b)=1$, $(a,a+b)=1$, and thus, $(ab,a+b)=1$. Since, $(ab,a+b)=1$, the only other possible factor that can divide both $2ab$ and $a+b$ is $2$. When we test, we see that if $a+b$ is even, it will be divisible by 2 and $(a^2 + b^2, a + b)=2$. If $a+b$ is odd, then $(a^2 + b^2, a + b)=1$.
\end{solution}
\end{document}