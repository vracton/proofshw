\documentclass{article}
\usepackage{enumitem}
\usepackage{fancyhdr}
\usepackage[a4paper, left=1in, right=1in, top=1in, bottom=1in]{geometry}
\usepackage{amsfonts}
\usepackage{amsmath}
\usepackage[many]{tcolorbox}
\usepackage{xcolor}
\usepackage{cancel}
\tcbuselibrary{skins}

\pagestyle{fancy}
\cfoot{}
\lhead{}

\definecolor{chinayellow}{rgb}{1, 0.3, 0.3}
\definecolor{chinared}{rgb}{0, 0.7, 0.7}

\chead{\Large\bf Number Theory HW \#9 - $\phi$ and Pseudoprimes}
\lhead{Page \thepage}
\rhead{Sonit Sahoo}
\setlength{\headheight}{18.0pt}

\newcommand\tagit{\addtocounter{equation}{1}\tag{\theequation}}

\newtcbtheorem[]{problem}{Problem}{
    enhanced,
    colback = chinayellow!5,
    colbacktitle = chinayellow!5,
    coltitle = black,
    boxrule = 0pt,
    frame hidden,
    borderline west = {0.6mm}{0mm}{chinayellow},
    fonttitle = \bfseries,
    before skip = 3ex,
    after skip = 0pt,
    sharp corners,
    rounded corners = northeast,
    breakable
}{problem}
\newtcbtheorem[]{solution}{}{
    enhanced,
    colback = chinared!5,
    coltitle = chinared!5,
    boxrule = 0pt,
    frame hidden,
    borderline west = {0.6mm}{0mm}{chinared},
    before skip = 0pt,
    after skip = 3ex,
    sharp corners,
    fonttitle = \tiny,
    rounded corners = southeast,
    attach title to upper = {},
    title after break = {},
    title = {},
    breakable
}{solution}

\begin{document}
\begin{problem}{}{}
    Factor $21,000,000$ and use the factorization to find $\phi(21,000,000)$.
\end{problem}
\begin{solution}{}{}
    \begin{align*}
        21000000 &= 21 \cdot 10^6 \\
        &= 3 \cdot 7 \cdot (2 \cdot 5)^6 \\
        &= 3 \cdot 7 \cdot 2^6 \cdot 5^6 \\
        &= 2^6 \cdot 3^1 \cdot 5^6 \cdot 7^1
    \end{align*}
    Thus, we have
    \begin{align*}
        \phi(21000000) &= 21000000 (1 - \frac{1}{2})\left(1 - \frac{1}{3}\right)\left(1 - \frac{1}{5}\right)\left(1 - \frac{1}{7}\right) \\
        &= 21000000 \cdot \frac{1}{2} \cdot \frac{2}{3} \cdot \frac{4}{5} \cdot \frac{6}{7} \\
        &= 21000000 \cdot \frac{1}{\cancel{2}} \cdot \frac{\cancel{2}}{\cancel{3}} \cdot \frac{4}{5} \cdot \frac{\cancel{6}^{\displaystyle{\hspace{0.1cm}2}}}{7} \\
        &= 21000000 \cdot \frac{8}{35}\\
        &= 4800000
    \end{align*}
\end{solution}

\begin{problem}{}{}
    Calculate $17^{17}$ and $35^{35}$ modulo 48.
\end{problem}
\begin{solution}{}{}
    $(17, 48) = 1$ and $(35, 48) = 1$, so it seems that we can use Euler's theorem! $48 = 2^4 \cdot 3$.
    \begin{align*}
        \phi(48) &= 48 \left(1 - \frac{1}{2}\right)\left(1 - \frac{1}{3}\right) \\
        &= 48 \cdot \frac{1}{2} \cdot \frac{2}{3} \\
        &= 48 \cdot \frac{1}{\cancel{2}} \cdot \frac{\cancel{2}}{3} \\
        &=16
    \end{align*}
    Thus, $17^{16}\equiv35^{16}\equiv1\pmod{48}$.
    \begin{align*}
        17^{17}&\equiv17^{16+1}\pmod{48} \\
        &\equiv17^{16}\cdot17^1\pmod{48} \\
        &\equiv1\cdot17\pmod{48} \\
        &\equiv\mathbf{17\pmod{48}}
    \end{align*}
    \begin{align*}
        35^{35}&\equiv35^{16\cdot2+3}\pmod{48} \\
        &\equiv(35^{16})^2\cdot35^3\pmod{48} \\
        &\equiv1^2\cdot42875\pmod{48} \\
        &\equiv42875\pmod{48} \\
        &\equiv\mathbf{11\pmod{48}}
    \end{align*}
\end{solution}

\begin{problem}{}{}
    Calculate $650^{650} \pmod{240}$.
\end{problem}
\begin{solution}{}{}
    $650=240*2+170$, so $650^{650}\equiv(240\cdot2+170)^{650}\equiv170^{650}\pmod{240}$. $240=16\cdot15$. $(16,15)=1$, so we can use the Chinese Remainder Theorem. First, we need to find $170^{650}\pmod{16}$.
    \begin{align*}
        170^{650}&\equiv(16\cdot10+10)^{650}\pmod{16} \\
        &\equiv10^{650}\pmod{16} \\
    \end{align*}
    $16=2^4$, so if we multiply by $5^4$, we get $10^4$. Thus, $16|10^4$, so $10^{4}\equiv0\pmod{16}$.
    \begin{align*}
        10^{650}&\equiv10^{4\cdot162+2}\pmod{16} \\
        &\equiv(10^4)^{162}\cdot10^2\pmod{16} \\
        &\equiv0^{162}\cdot10^2\pmod{16} \\
        &\equiv0\pmod{16}
    \end{align*}
    So, $170^{650}\equiv0\pmod{16}$. Now, let's calculate $170^{650}\pmod{15}$.
    \begin{align*}
        170^{650}&\equiv(15\cdot11+5)^{650}\pmod{15} \\
        &\equiv5^{650}\pmod{15} \\
    \end{align*}
    Now, let's make a list of $5^n\pmod{15}$.
    \begin{align*}
        5^1 \equiv 5 &\pmod{15} \\
        5^2 \equiv 25 &\equiv 10 \pmod{15} \\
        5^3 \equiv 50 &\equiv 5 \pmod{15} \\
        5^4 \equiv 25 &\equiv 10 \pmod{15} \\
        5^5 \equiv 50 &\equiv 5 \pmod{15} \\
        \vdots&
    \end{align*}
    As we can see, this is a cycle between 5 and 10. 5 to an even power $\equiv 10\pmod{15}$ and 5 to an odd power $\equiv 5\pmod{15}$. Since $650$ is even, $5^{650}\pmod{15}\equiv{10}\pmod{15}$. So, we have two congruences:
    \begin{align*}
        170^{650} &\equiv 0 \pmod{16} \\
        170^{650} &\equiv 10 \pmod{15}
    \end{align*}
    Now, let's use the Chinese Remainder Theorem with $x=170^{650}$.
    \begin{align*}
        x&\equiv 0\pmod{16} \tagit \\
        x&=16k_1 \text{ for } k_1\in\mathbb{Z} \\
        16k_1 &\equiv 10 \pmod{15} \tagit \\
        k_1 &\equiv 10 \pmod{15} \\
        k_1 &= 10+15k_2 \text{ for } k_2\in\mathbb{Z}\\
        x&=16(10+15k_2) \\
        x &= 160+240k_2 \\
        x &\equiv 160 \pmod{240} \tagit
    \end{align*}
    Thus, $650^{650} \equiv 160 \pmod{240}$.
\end{solution}

\begin{problem}{}{}
    The number 137 is prime. What are the possibilities for $a^{68}$ to be congruent to, modulo 137.
\end{problem}
\begin{solution}{}{}
    There are two cases for $a$: it is coprime to 137, or it is not. If $a$ is not coprime to 137, then they share a factor $\neq1$. However, the only other factor of 137 is 137, so $137|a$. Thus, in this case, $a^{68}\equiv0\pmod{137}$. If $a$ is coprime to 137, then we can use Fermat's little theorem: $a^{136}\equiv1\pmod{137}$. Let $a^{68}\equiv{x}\pmod{137}$. We notice that $a^{136}=a^{68\cdot2}\equiv a^{68}\cdot a^{68}$. Since modulus is multiplicative, $a^{68}\cdot a^{68}\equiv a^{136}\pmod{137}$. Thus, $x^2\equiv1\pmod{137}$. This means that either $x\equiv1\pmod{137}$ or $x\equiv-1\pmod{137}$ when $a$ is coprime to 137. So, the possibilities for $a^{68}$ to be congruent to modulo 137 are $0$, $1$, and $-1$.
\end{solution}

\begin{problem}{}{}
    Verify that 25 is a base-7 pseudoprime.
\end{problem}
\begin{solution}{}{}
    We need to verify that $7^{25} \equiv 7 \pmod{25}$. Since $(7,25) = 1$, we can use Euler's theorem. First, we need to find $\phi(25)$.
    \begin{align*}
        25 &= 5^2 \\
        \phi(25) &= 25 \left(1 - \frac{1}{5}\right) \\
        &= 25 \cdot \frac{4}{5} \\
        &= 20
    \end{align*}
    Thus, $7^{20} \equiv 1 \pmod{25}$. Let's make a list of the repeated squaring method.
    \begin{align*}
        7^1 &\equiv 7 \pmod{25} \\
        7^2 &\equiv 7^2 \equiv 49 \equiv 24 \pmod{25} \\
        7^4 &\equiv 24^2 \cdot 576 \equiv 1 \equiv 576 \equiv 1 \pmod{25}
    \end{align*}
    Now, lets calculate $7^{25}\pmod{25}$.
    \begin{align*}
        7^{25} &\equiv 7^{20+4+1} \pmod{25} \\
        &\equiv 7^{20} \cdot 7^4 \cdot 7^1 \pmod{25} \\
        &\equiv 1 \cdot 1 \cdot 7 \pmod{25} \\
        &\equiv 7 \pmod{25}
    \end{align*}
    Thus, $7^{25} \equiv 7 \pmod{25}$, so we have shown that 25 is a base-7 pseudoprime.
\end{solution}

\begin{problem}{}{}
    Verify that $2047=23\cdot89$ is a base-2 pseudoprime.
\end{problem}
\begin{solution}{}{}
    We need to verify that $2^{2047} \equiv 2 \pmod{2047}$. Since $(2,2047) = 1$, we can use Euler's theorem. First, we need to find $\phi(2047)$. We are given that $2047 = 23 \cdot 89$.
    \begin{align*}
        \phi(2047) &= 2047 \left(1 - \frac{1}{23}\right)\left(1 - \frac{1}{89}\right) \\
        &= 2047 \cdot \frac{22}{23} \cdot \frac{88}{89} \\
        &= 22\cdot88 \\
        &= 1936
    \end{align*}
    So, $2^{1936} \equiv 1 \pmod{2047}$. Now, let's make a list with the repeated squaring method.
    \begin{align*}
        2^1 &\equiv 2 \pmod{2047} \\
        2^2 &\equiv 4 \pmod{2047} \\
        2^4 &\equiv 16 \pmod{2047} \\
        2^8 &\equiv 256 \pmod{2047} \\
        2^{16} &\equiv 256^2\equiv 65536 \equiv 32 \pmod{2047} \\
        2^{32} &\equiv 32^2 \equiv 1024 \pmod{2047} \\
        2^{64} &\equiv 1024^2 \equiv 1048576 \equiv 512 \pmod{2047} \\
    \end{align*}
    Now, let's calculate $2^{2047}\pmod{2047}$.
    \begin{align*}
        2^{2047} &\equiv 2^{1936+64+32+8+4+2+1} \pmod{2047} \\
        &\equiv 2^{1936} \cdot 2^{64} \cdot 2^{32} \cdot 2^8 \cdot 2^4 \cdot 2^2 \cdot 2^1 \pmod{2047} \\
        &\equiv 1 \cdot 512 \cdot 1024 \cdot 256 \cdot 16 \cdot 4 \cdot 2 \pmod{2047} \\
        &\equiv 2\cdot 256\cdot 1024 \cdot 256 \cdot 16 \cdot 4 \cdot 2 \pmod{2047} \\
        &\equiv 1024\cdot(256\cdot 256)\cdot(16\cdot4\cdot2\cdot2)\pmod{2047} \\
        &\equiv 4\cdot256\cdot256^2\cdot16^2\pmod{2407} \\
        &\equiv 4\cdot256\cdot32\cdot256\pmod{2047} \\
        &\equiv 4\cdot256^2\cdot32\pmod{2047} \\
        &\equiv 4\cdot32\cdot32\pmod{2047} \\
        &\equiv 4\cdot32^2\pmod{2047} \\
        &\equiv 4\cdot1024\pmod{2047} \\
        &\equiv 4096\pmod{2047} \\
        &\equiv 2047\cdot2+2pmod{2047} \\
        &\equiv 2\pmod{2047}
    \end{align*}
    Thus, $2^{2047} \equiv 2 \pmod{2047}$, so we have shown that 247 is a base-2 pseudoprime.
\end{solution}

\end{document}